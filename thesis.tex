
% ----------------------------------------------------------------------
%                   LATEX TEMPLATE FOR PhD THESIS
% ----------------------------------------------------------------------

% based on Harish Bhanderi's PhD/MPhil template, then Uni Cambridge
% http://www-h.eng.cam.ac.uk/help/tpl/textprocessing/ThesisStyle/
% corrected and extended in 2007 by Jakob Suckale, then MPI-CBG PhD programme
% and made available through OpenWetWare.org - the free biology wiki


%: Style file for Latex
% Most style definitions are in the external file PhDthesisPSnPDF.
% In this template package, it can be found in ./Latex/Classes/
\documentclass[twoside,11pt]{Latex/Classes/PhDthesisPSnPDF}


%: Macro file for Latex
% Macros help you summarise frequently repeated Latex commands.
% Here, they are placed in an external file /Latex/Macros/MacroFile1.tex
% An macro that you may use frequently is the figuremacro (see introduction.tex)
\usepackage{mathtools}

\usepackage{color}

\definecolor{pblue}{rgb}{0.13,0.13,1}
\definecolor{pgreen}{rgb}{0,0.5,0}
\definecolor{pred}{rgb}{0.9,0,0}
\definecolor{pgrey}{rgb}{0.46,0.45,0.48}
\usepackage{listings}

\lstdefinestyle{customjava}{
  breakatwhitespace=false,         % sets if automatic breaks should only happen at whitespace
  breaklines=true,                 % sets automatic line breaking
  captionpos=b,                    % sets the caption-position to bottom
  extendedchars=true,              % lets you use non-ASCII characters; for 8-bits encodings only, does not work with UTF-8
  frame=single,                    % adds a frame around the code
  language=Java,                 % the language of the code
  keywordstyle=\bf,
  showspaces=false,                % show spaces everywhere adding particular underscores; it overrides 'showstringspaces'
  showstringspaces=false,          % underline spaces within strings only
  showtabs=false,                  % show tabs within strings adding particular underscores
  tabsize=2,                       % sets default tabsize to 2 spaces
  basicstyle=\footnotesize\ttfamily,
  commentstyle=\color{pgreen},
  keywordstyle=\color{pblue},
  stringstyle=\color{pred},
  moredelim=[il][\textcolor{pgrey}]{$$},
  moredelim=[is][\textcolor{pgrey}]{\%\%}{\%\%}
}

\usepackage{xcolor}
\colorlet{punct}{red!60!black}
\definecolor{background}{HTML}{EEEEEE}
\definecolor{delim}{RGB}{20,105,176}
\colorlet{numb}{magenta!60!black}

\lstdefinelanguage{json}{
    basicstyle=\normalfont\ttfamily,
    numbers=left,
    numberstyle=\scriptsize,
    stepnumber=1,
    numbersep=8pt,
    showstringspaces=false,
    breaklines=true,
    frame=lines,
    backgroundcolor=\color{background},
    literate=
     *{0}{{{\color{numb}0}}}{1}
      {1}{{{\color{numb}1}}}{1}
      {2}{{{\color{numb}2}}}{1}
      {3}{{{\color{numb}3}}}{1}
      {4}{{{\color{numb}4}}}{1}
      {5}{{{\color{numb}5}}}{1}
      {6}{{{\color{numb}6}}}{1}
      {7}{{{\color{numb}7}}}{1}
      {8}{{{\color{numb}8}}}{1}
      {9}{{{\color{numb}9}}}{1}
      {:}{{{\color{punct}{:}}}}{1}
      {,}{{{\color{punct}{,}}}}{1}
      {\{}{{{\color{delim}{\{}}}}{1}
      {\}}{{{\color{delim}{\}}}}}{1}
      {[}{{{\color{delim}{[}}}}{1}
      {]}{{{\color{delim}{]}}}}{1},
}

\include{Latex/Macros/MacroFile1}



%: ----------------------------------------------------------------------
%:                  TITLE PAGE: name, degree,..
% ----------------------------------------------------------------------
% below is to generate the title page with crest and author name

%if output to PDF then put the following in PDF header
\ifpdf  
    \pdfinfo { /Title  (Enhancing 3D reconstruction with Mobile Sensors Data)
               /Creator (TeX)
               /Producer (pdfTeX)
               /Author (Krzysztof Wróbel krzysztof@velocicaptor.com)
               /CreationDate (D:YYYYMMDDhhmmss)  %format D:YYYYMMDDhhmmss
               /ModDate (D:YYYYMMDDhhmm)
               /Subject (xyz)
               /Keywords (3D reconstruction, Augmented Relaity, AR, Sensor, Accelerometer, Gyroscope, Sensor Fusion, Bundle Adjustment, Pose Estimation) }
    \pdfcatalog { /PageMode (/UseOutlines)
                  /OpenAction (fitbh)  }
\fi

%\include{title}

\title{Enhancing 3D reconstruction using Mobile Sensors Data}



% ----------------------------------------------------------------------
% The section below defines www links/email for author and institutions
% They will appear on the title page of the PDF and can be clicked
\ifpdf
  \author{\href{mailto:krzysztof@velocicaptor.com}{Krzysztof Wróbel}}
%  \cityofbirth{born in XYZ} % uncomment this if your university requires this
%  % If city of birth is required, also uncomment 2 sections in PhDthesisPSnPDF
%  % Just search for the "city" and you'll find them. TODO
  \collegeordept{\href{http://www.something.net}{CollegeOrDepartment}}
  \university{\href{http://www.something.net}{University}}

  % The crest is a graphics file of the logo of your research institution.
  % Place it in ./0_frontmatter/figures and specify the width
  \crest{\includegraphics[width=4cm]{logo}}
  
% If you are not creating a PDF then use the following. The default is PDF.
\else
  \author{Krzysztof Wróbel}
%  \cityofbirth{born in XYZ}
  \collegeordept{CollegeOrDept}
  \university{University}
  \crest{\includegraphics[width=4cm]{logo}}
\fi

%\renewcommand{\submittedtext}{change the default text here if needed}
\degree{Philosophi\ae Doctor (PhD), DPhil,..}
\degreedate{year month}


% ----------------------------------------------------------------------
       
% turn of those nasty overfull and underfull hboxes
\hbadness=10000
\hfuzz=50pt


%: --------------------------------------------------------------
%:                  FRONT MATTER: dedications, abstract,..
% --------------------------------------------------------------

\begin{document}

%\language{english}

% sets line spacing
\renewcommand\baselinestretch{1.2}
\baselineskip=18pt plus1pt


%: ----------------------- generate cover page ------------------------

\maketitle  % command to print the title page with above variables


%: ----------------------- cover page back side ------------------------
% Your research institution may require reviewer names, etc.
% This cover back side is required by Dresden Med Fac; uncomment if needed.

\newpage
\vspace{10mm}
1. Reviewer: Name

\vspace{10mm}
2. Reviewer: 

\vspace{20mm}
Day of the defense:

\vspace{20mm}
\hspace{70mm}Signature from head of PhD committee:



%: ----------------------- abstract ------------------------

% Your institution may have specific regulations if you need an abstract and where it is to be placed in the document. The default here is just after title.


% Thesis Abstract -----------------------------------------------------


%\begin{abstractslong}    %uncommenting this line, gives a different abstract heading
\begin{abstracts}        %this creates the heading for the abstract page

The main subject of this thesis addresses the possibility of enhancing the existing 3D reconstruction algorithms with additional Android sensor fusion data. The primary fundamentals behind the reconstruction techniques and the problems related to them are explained in order to introduce the reader to these topics. Related works are discussed in order to explain the strengths and weaknesses of the state-of-art reconstruction approches. The proposed enhancements using initial rotation in the standard 8-point and 5-point algorithms are explained in detail. Both of them allow for focusing on correcting rotation matrix error instead of calculating the rotation itself. This, in turn, serves for reducing ambiguity of the decomposition of essential matrix to the relative rotation and translation of the cameras. The primary idea behind the implementation of the 3-point algorithm for translation is also described. The most important aspects of the implemented $"Sensor Enhanced Image Camera"$ Android application and $"Enhanced 3D Reconstructer"$ were discussed.
The three proposed initial pair reconstruction methods were evaluated, which showed that each algorithm is able to improve certain asspects of the 3D reconstruction. Different reconstruction strategies were evaluated. As a result it was discovered that the enhanced structure computation can be faster and more accurate than the standard methodology. Using initial rotation and translation pose estimation results in significantly faster convergence and better error reduction with Bundle Adjustment. This thesis covers only some notions of the Structure from Motion, therefore plans for future development were  established.

\end{abstracts}
\clearpage
\renewcommand{\abstractname}{Zusammenfassung}
\begin{abstracts}        %this creates the heading for the abstract page

Das Hauptthema dieser Arbeit befasst sich mit der M{\"o}glichkeit der Verbesserung der vorhandenen 3D-Rekonstruktionsalgorithmen mit zus{\"a}tzlichen Android Sensorfusion Daten. Die wichtigsten Grundlagen hinter den Rekonstruktionstechniken und die damit verbundenen Probleme sind, um den Leser zu diesen Themen vorstellen erl{\"a}utert. {\"A}hnliche Arbeiten sind erforderlich, um die Festigkeiten und Schw{\"a}chen der Stand-der-Technik Rekonstruktion Ansätze erkl{\"a}ren diskutiert. Die vorgeschlagenen Verbesserungen mit Anfangsrotation in den Standard 8-Punkte und 5-Punkte-Algorithmen werden im Detail erkl{\"a}rt. Beide von ihnen k{\"o}nnen zur Fokussierung auf die Rotationsmatrix Fehler Korrektur anstelle der Berechnung des Rotation selbst. Dies wiederum serviert zur Reduzierung der Mehrdeutigkeit des Essential-Matrix Zersetzungs zu der relativen Rotation und Translation den Kameras. Die Hauptidee hinter der Umsetzung der 3-Punkt-Algorithmus f{\"u}r die Translation ist ebenfalls beschrieben. Die wichtigsten Aspekte die implementierten $"Sensor Verbesserte Bildkamera "$ Android-Anwendung und $ "Verbesserte 3D Reconstructer "$  wurden diskutiert.
Die drei vorgeschlagenen ersten Paar Rekonstruktionsmethoden wurden bewertet, die zeigten, dass jeder Algorithmus kann bestimmte asspects der 3D-Rekonstruktion zu verbessern. Verschiedene Rekonstruktionsstrategien wurden bewertet. Als Ergebnis wurde festgestellt, dass die verbesserte Struktur Berechnung kann schneller und genauer als die Standardmethode sein. Mit  anfänglichen Rotation und Translation  Posenschätzung resultiert in signifikant schneller Konvergenz und besseres Fehlerreduktion mit B{\"u}ndelausgleichung. Diese Diplomarbeit umfasst nur einige Vorstellungen von der Struktur aus Bewegung, daher plant f{\"u}r die zuk{\"u}nftige Entwicklung entstanden.

\end{abstracts}
%\end{abstractlongs}


% ---------------------------------------------------------------------- 


% The original template provides and abstractseparate environment, if your institution requires them to be separate. I think it's easier to print the abstract from the complete thesis by restricting printing to the relevant page.
% \begin{abstractseparate}
%   
% Thesis Abstract -----------------------------------------------------


%\begin{abstractslong}    %uncommenting this line, gives a different abstract heading
\begin{abstracts}        %this creates the heading for the abstract page

The main subject of this thesis addresses the possibility of enhancing the existing 3D reconstruction algorithms with additional Android sensor fusion data. The primary fundamentals behind the reconstruction techniques and the problems related to them are explained in order to introduce the reader to these topics. Related works are discussed in order to explain the strengths and weaknesses of the state-of-art reconstruction approches. The proposed enhancements using initial rotation in the standard 8-point and 5-point algorithms are explained in detail. Both of them allow for focusing on correcting rotation matrix error instead of calculating the rotation itself. This, in turn, serves for reducing ambiguity of the decomposition of essential matrix to the relative rotation and translation of the cameras. The primary idea behind the implementation of the 3-point algorithm for translation is also described. The most important aspects of the implemented $"Sensor Enhanced Image Camera"$ Android application and $"Enhanced 3D Reconstructer"$ were discussed.
The three proposed initial pair reconstruction methods were evaluated, which showed that each algorithm is able to improve certain asspects of the 3D reconstruction. Different reconstruction strategies were evaluated. As a result it was discovered that the enhanced structure computation can be faster and more accurate than the standard methodology. Using initial rotation and translation pose estimation results in significantly faster convergence and better error reduction with Bundle Adjustment. This thesis covers only some notions of the Structure from Motion, therefore plans for future development were  established.

\end{abstracts}
\clearpage
\renewcommand{\abstractname}{Zusammenfassung}
\begin{abstracts}        %this creates the heading for the abstract page

Das Hauptthema dieser Arbeit befasst sich mit der M{\"o}glichkeit der Verbesserung der vorhandenen 3D-Rekonstruktionsalgorithmen mit zus{\"a}tzlichen Android Sensorfusion Daten. Die wichtigsten Grundlagen hinter den Rekonstruktionstechniken und die damit verbundenen Probleme sind, um den Leser zu diesen Themen vorstellen erl{\"a}utert. {\"A}hnliche Arbeiten sind erforderlich, um die Festigkeiten und Schw{\"a}chen der Stand-der-Technik Rekonstruktion Ansätze erkl{\"a}ren diskutiert. Die vorgeschlagenen Verbesserungen mit Anfangsrotation in den Standard 8-Punkte und 5-Punkte-Algorithmen werden im Detail erkl{\"a}rt. Beide von ihnen k{\"o}nnen zur Fokussierung auf die Rotationsmatrix Fehler Korrektur anstelle der Berechnung des Rotation selbst. Dies wiederum serviert zur Reduzierung der Mehrdeutigkeit des Essential-Matrix Zersetzungs zu der relativen Rotation und Translation den Kameras. Die Hauptidee hinter der Umsetzung der 3-Punkt-Algorithmus f{\"u}r die Translation ist ebenfalls beschrieben. Die wichtigsten Aspekte die implementierten $"Sensor Verbesserte Bildkamera "$ Android-Anwendung und $ "Verbesserte 3D Reconstructer "$  wurden diskutiert.
Die drei vorgeschlagenen ersten Paar Rekonstruktionsmethoden wurden bewertet, die zeigten, dass jeder Algorithmus kann bestimmte asspects der 3D-Rekonstruktion zu verbessern. Verschiedene Rekonstruktionsstrategien wurden bewertet. Als Ergebnis wurde festgestellt, dass die verbesserte Struktur Berechnung kann schneller und genauer als die Standardmethode sein. Mit  anfänglichen Rotation und Translation  Posenschätzung resultiert in signifikant schneller Konvergenz und besseres Fehlerreduktion mit B{\"u}ndelausgleichung. Diese Diplomarbeit umfasst nur einige Vorstellungen von der Struktur aus Bewegung, daher plant f{\"u}r die zuk{\"u}nftige Entwicklung entstanden.

\end{abstracts}
%\end{abstractlongs}


% ---------------------------------------------------------------------- 

% \end{abstractseparate}


%: ----------------------- tie in front matter ------------------------

\frontmatter
\include{0_frontmatter/dedication}
\include{0_frontmatter/acknowledgement}


%: ----------------------- contents ------------------------

\setcounter{secnumdepth}{3} % organisational level that receives a numbers
\setcounter{tocdepth}{3}    % print table of contents for level 3
\tableofcontents            % print the table of contents
% levels are: 0 - chapter, 1 - section, 2 - subsection, 3 - subsection


%: ----------------------- list of figures/tables ------------------------

\listoffigures	% print list of figures

\listoftables  % print list of tables


%: ----------------------- glossary ------------------------

% Tie in external source file for definitions: /0_frontmatter/glossary.tex
% Glossary entries can also be defined in the main text. See glossary.tex
% this file is called up by thesis.tex
% content in this file will be fed into the main document

% Glossary entries are defined with the command \nomenclature{1}{2}
% 1 = Entry name, e.g. abbreviation; 2 = Explanation
% You can place all explanations in this separate file or declare them in the middle of the text. Either way they will be collected in the glossary.

% required to print nomenclature name to page header
%\markboth{\MakeUppercase{\nomname}}{\MakeUppercase{\nomname}}


% ----------------------- contents from here ------------------------
\newglossaryentry{fundamentalMatrix}
{
name={Fundamental Matrix},
description = {Todo}
} 
\newglossaryentry{rgb}
{
name={RGB Camera},
description = {Camera, which deliver image consting of 3 basic colors: red, green and blue}
} 
\newglossaryentry{ir}
{
name={IR depth-finding Camera},
description = {Camera, which deliver information about depth in image}
} 
\newacronym{ba}{BA}{Bundle Adjustment} 
\newacronym{ba}{RGB Camera}{Bundle Adjustment} 
\newacronym{ba}{IR}{Bundle Adjustment} 
\newacronym{ransac}{RANSAC}{RANdom SAmple Consensus} 
\newacronym{roll}{Pitch}{Rotation around the x axis.} 
\newacronym{pitch}{Azimuth}{Rotation around the y axis.} 
\newacronym{yaw}{Roll}{Rotation around the z axis.} 
\newacronym{svd}{SVD}{Singular Value Decomposition} 
\newacronym{fov}{FoV}{Field of View}
\newacronym{baseLine}{Base line}{The distance between the left and the right camera in a stereo pair.} 
\newacronym{aa9}{Essential Matrix}{TODO} 
\newacronym{a9a}{Pose Estimation}{TODO} 
\newacronym{aa99}{Internal camera parameters}{TODO} 
\newacronym{aa11}{Extrasinc??? camera parameters }{TODO} 
\newacronym{aa12}{Rotation Matrix}{TODO} 
\newacronym{aa13}{Quaternion}{TODO} 
\newacronym{aa14}{Epipolar lines}{TODO} 
\newacronym{aa15}{SIFT}{TODO} 


 

\begin{multicols}{2} % \begin{multicols}{#columns}[header text][space]
\begin{footnotesize} % scriptsize(7) < footnotesize(8) < small (9) < normal (10)

\printnomenclature[1.5cm] % [] = distance between entry and description
\label{nom} % target name for links to glossary

\end{footnotesize}
\end{multicols}



%: --------------------------------------------------------------
%:                  MAIN DOCUMENT SECTION
% --------------------------------------------------------------

% the main text starts here with the introduction, 1st chapter,...
\mainmatter

\renewcommand{\chaptername}{} % uncomment to print only "1" not "Chapter 1"


%: ----------------------- subdocuments ------------------------

% Parts of the thesis are included below. Rename the files as required.
% But take care that the paths match. You can also change the order of appearance by moving the include commands.

% this file is called up by thesis.tex
% content in this file will be fed into the main document

%: ----------------------- introduction file header -----------------------
\chapter{Introduction}
% the code below specifies where the figures are stored
\ifpdf
    \graphicspath{{1_introduction/figures/PNG/}{1_introduction/figures/PDF/}{1_introduction/figures/}}
\else
    \graphicspath{{1_introduction/figures/EPS/}{1_introduction/figures/}}
\fi

% ----------------------------------------------------------------------
%: ----------------------- introduction content ----------------------- 
% ----------------------------------------------------------------------

Mobile and wearable devices are becoming more and more popular. Modern smartphones despite having extremely good camera's also use advanced sensor's, like accelerometers, gyroscope, magnetometer, barometer etc.. There is also a big need and growing market of Augmented Reality (AR) and Virtual Reality(VR). That is why image analysis and recognition, as well as 3D reconstruction techniques are really hot topic. Unfortunately algorithms that support these techniques are very time and memory consuming, that is why it is really hard to run them on mobile devices, which have many limitations in terms of CPU speed and RAM memory capacity.

\section{Purpose of this thesis} % section headings are printed smaller than chapter names
Author of this document will present the reader with an overview of the idea of 3D reconstruction techniques. This thesis also inculdes brief description of related research conducted in this area. After short analysis of efficiency, accuracy and common problems of few chosen algorithms this thesis will propose their enhacment using data acquired from Android sensor fusion. Two application was created in order to verify and test proposed algorithms and reconstruction strategies. Both Android and Desktop applications will be discussed in important implementation asspects. This thesis also explains the reader meaning of testing results. At the end conclusions and encountered errors are discussed. Finally futur plans for conducted research is presented.
\section{Scope}
The author researched, how fusion of accelerometer, gyroscope and magnetometer can be used in order to improve fundamental, essential matrix calculations and also relative pose estimation. Unfortunately raw data sensors are really noisy and therefore not reliable enough to relay solely on them to enhance reconstruction. Heuristic measurment of translation used for enhancment using linear-acceleration was also concepted and implemented by author. In this research introduced are  initial rotation enhanced 8-point and 5-point reconstruction algorithms. Also alternative 3-point algorithm for translation estimation is also discussed and implemented.
This thesis deals only with initial pair images reconstruction and relative pose estimation for Structure from Motion computation. It is not researched, what are the best methodology for corresponding feature computation, but other possibilities of point matching are discussed. Finally it is reserched, how improved versions of standard 8-point and 5-point algorithms influence convergence speed and error reduction of Bundle Adjustment.
\section{Initial assumptions}
The general process of 3-D reconstruction is quite broad, that's why the author of the thesis focus only on certain aspects of this topic. That is why author did not write all algorithms from the scratch, but built his versions on top of OpenCV library and open-source project called "Relative Pose Esitmation" created by TODO . In terms of sensor fusion, currently state of art approach can be found on Android platform. 3D reconstruction process will be processed in seperate desktop application.
\section{Thesis Outline}
In \textbf{Chapter 2} fundamental technology behind 3D reconstruction and Structure from Motion was discussed. Also sensors used to perform sensor fusion on Android platform were briefly introduced.
\\*
In \textbf{Chapter 3} brief overview of related research was made. Most interesting discoveries, which lead to conception of proposed methods were also described. 
\\*
In \textbf{Chapter 4} teoretical concept of proposed 8-point and 5-point rotation enhanced algorithms were presented. Also alternative approach to 3-point translation estimation was made. Finally author proposed few most promising Structure from motion reconstruction strategies.
\\*
In \textbf{Chapter 5} most important implementation asspects of rotation and translation measurements were discussed. Also enhancements modifications to standard 8-point and 5-point algorithms were discussed. Also 3-point translation estimation approach and relative pose ehancements implementation was discussed.
\\*
In \textbf{Chapter 6} evaluation of proposed initial pair reconstruction methods and strategies was presented and discussed.
\\*
In \textbf{Chapter 7} conslusions observed during master thesis project period, as well as encounterd problems were discussed. Thesis ends with further development plans establishment.

% ----------------------------------------------------------------------



	% background information
\ifpdf
    \graphicspath{{figures/}{figures/comparisons}}
\else
    \graphicspath{{figures/}{figures/comparisons}}
\fi

% this file is called up by thesis.tex
% content in this file will be fed into the main document

\chapter{Fundamentals} % top level followed by section, subsection
This chapter explains the basic theory behind 3D reconstruction and Structure from Motion. All information referred to herein can be found in \cite{HartleyMultipleView}. It also includes a brief overview of sensors, which can be used to calculate global or relative orientation of the smartphone camera. It is also explained, why it is necessary to use accelerometer, gyroscope and magnetometer together by combing them into Sensor Fusion. 

% ----------------------- contents from here ------------------------
\section{3D reconstruction}
In general 3D reconstruction is the process of automatic creation of 3 dimensional objects models from images. There are many possibilities of performing 3D reconstruction, starting from two-view reconstruction, where only two images are used to multiple-view reconstruction, where image sequence is bigger than 2. Reconstruction can be performed either with a single hand-held camera from images sequence or simultaneously working and synchronised multiple cameras. It can be done either on spot in real time or later for instance in laboratory. As few as two pictures taken from different angles of a single object are sufficient to perform 3D model generation. In general reconstruction process consists of the following steps:
\setlist[2]{noitemsep} % sets the itemsep and parsep for all level two lists to 0
\setenumerate{noitemsep} % sets no itemsep for enumerate lists only
\begin{enumerate}
\item \textbf{Image Acquisition}, where images frames are acquired 
\item \textbf{Feature extraction and corresponding points matching}, where distinctive features are extracted from the images and compared
\item \textbf{Fundamental \& Essential Matrices computation}, where matrices meeting the requirements of basic epipolar geometry are calculated
\item \textbf{Camera parameters estimation}, where external and internal camera parameters, like translation and rotation are estimated
\item \textbf{Triangulation}, where camera projection matrices are composed and used in order to calculate 3D cloud points
\end{enumerate}
\subsection{Camera model}
To get reader acquainted with nomenclature used when reading about camera related operations following section was written. In general two coordinations system are related to camera model, which can be seen in Figure \ref{fig:camera_model}:
\begin{enumerate} 
\item The external coordinate system (denoted here with a subscript \textbf{W} for world) which is independent of placement and parameters of the camera.
\item The camera coordinate system (denoted by \textbf{C}, for camera).
\end{enumerate}
Relation between those two coordinate systems is expressed by translation matrix \textbf{T} and rotation matrix \textbf{R}. These two together are often referred as external camera parameters.
\begin{figure}[h!]
    \centering
    \includegraphics[width=0.8\textwidth]{camera_model}
    \caption{Model of the perspective camera with two coordinate systems: external W and
internal K. Image taken from \cite{Cyganek3dVision}.}
    \label{fig:camera_model}
\end{figure}
Internal camera parameters are expressed by the following matrix:
\begin{equation}
\begin{array}{lcl}
\textbf{K} & = &
\begin{bmatrix}
h_{x} & 0 & o_{x} \\ 
0 & h_{y} & o_{y} \\ 
0 & 0 & 1
\end{bmatrix}
\end{array}
\end{equation}
where $h_{x}$ and $h_{y}$ represent the focal length of the camera expressed in pixel dimensions in the x and y direction respectively. Similarly, point \textbf{o} = $(o_{x},o_{y})$ is the principal point of camera in pixel dimensions. These parameters need to be calculated only once for each camera model. Once they are known, camera can be described as calibrated. 
Calibration of cameras can be done with special reference boards of the known dimensions and characteristics. One of such boards (Figure ) as well as sample code that allows to calibrate camera is available on OpenCV website \cite{website:cameraCalibration}.
\begin{figure}[h!]
    \centering
    \includegraphics[width=0.8\textwidth]{opencv_camera_calib}
    \caption{OpenCV camera calibration board. Image taken from \cite{website:cameraCalibration}.}
    \label{fig:camera_model}
\end{figure}
Knowledge of external and internal camera parameters allows us to switch between camera and world coordinates system and is necessary for proper 3D model reconstruction.
\subsection{Feature extraction and corresponding points matching}
Usually, each image used in reconstruction has to be analysed in order for the distinctive features to be found. Afterwards all features in images are compared in order to find corresponding matches between them (Figure \ref{fig:correspondingMatches}).
\begin{figure}[h]
    \centering
    \includegraphics[width=0.8\textwidth]{correspondingMatching}
    \caption{Corresponding matches in ''Warsaw Gallery Dataset'' found for SIFT features and brute-force matcher using written by author desktop application}
    \label{fig:correspondingMatches}
\end{figure}
There are multiple features detectors and extractors available for use \cite{website:featureDetection}. Some of them are more suitable for edge detection, while others are best used for corner or blob based feature detection. One of the most popular and robust feature detection method is scale-invariant feature transform (SIFT) \cite{website:SIFT}. The use of these descriptors allows for detecting local features in images and describing them with special metrics, which are scale, rotation and translation invariant. Unfortunately even using such complex descriptors this process still is really hard and difficult. Algorithms that do these types of matching are usually O($n^2$) and therefore it takes a lot of time to find proper correspondences, especially when good object (image) resolution is needed. What is more even the most distinctive descriptors do not give global unique values and thus a lot of outliers can be produced during matching process (Figure \ref{fig:failMatching}). This is one of the first problems, which do not allow for perfectly accurate and error free 3D reconstruction.
\begin{figure}[ht!]
    \centering
    \includegraphics[width=0.8\textwidth]{failMatching}
    \caption[Outliers found during matching correspondences in two architecture pictures]{As shown in image example a lot of points in images can be improperly matched. This can influence further effectiveness of 3D reconstruction \cite{website:failMatching} }
    \label{fig:failMatching}
\end{figure}
\subsection{Epipolar geometry}
In Figure \ref{fig:epipolar_geometry} epipolar geometry schematic is shown. Baseline is the line connecting two cameras center points. Points in which baseline is crossing cameras image planes are called epipoles ($\textbf{e}_l$ and $\textbf{e}_r$). Lines that are going through epipoles and corresponding feature points $\textbf{p}_l$ and $\textbf{p}_r$ are called epipolar lines. Rays coming from cameras center points through corresponding feature points cross in position of 3D point \textbf{P} of photographed object.
\begin{figure}[!h]
    \centering
    \includegraphics[width=0.8\textwidth]{epipolar_geometry}
    \caption{Epipolar geometry schematic. Picture from \cite{Cyganek3dVision}. }
    \label{fig:epipolar_geometry}
\end{figure}
Figure \ref{fig:EpipolarGeometry} shows different perspective on epipolar lines and example of their visualisation in image planes. These lines cross the exactly same points in both images and can be used for dense feature matching since matches need to be searched exclusively in the surroundings of these lines. However they can be calculated only if essential or fundmental matrix is determined.
\begin{figure}[!h]
    \centering
    \includegraphics[width=0.8\textwidth]{EpipolarGeometry}
    \caption[Epipolar lines found in an image of a vase]{Epipolar lines found in an image of a vase \cite{HartleyMultipleView}. It is good to spot that all points that lay on corresponding epipolar lines and are visible in both images can be easily matched}
    \label{fig:EpipolarGeometry}
\end{figure} \\
\subsection{Fundamental \& Essential Matrix estimations}
Once proper feature matches are found in two images, it can be proven that there exists Fundamental matrix F for which the following equation is satisfied:
\begin{equation} \label{eq:fundamntalEquation}
\textbf{x}^{'T} \textbf{F} \textbf{x} = 0
\end{equation} 
where $x$ and ${x}^{'}$ are uncalibrated notions of points correspondence \cite{HartleyMultipleView}. It is known that solutions of this equation are highly sensitive to the occurrence of outliers. Usually, to make fundamental matrix estimations more accurate some outliers removing algorithms need to be used. One of the most robust approaches includes the use of of RANdom SAmple Consensus (RANSAC) \cite{website:ransac}.
\begin{figure}[h!]
    \centering
    \includegraphics[width=0.8\textwidth]{RANSACFitting}
    \caption[RANSAC fitting for 2D image]{RANSAC fitting for 2D image\cite{website:ransac}. In this example all red points are some existing outliers. Blue line is decided by iterative process of fitting line to randomly chosen subset of all points. When enough percentage of all points lies in the neighbourhood of such fitted line(blue subset), process stops and unnecessary points (red points) can be removed for further processing.}
    \label{fig:RANSACFitting}
\end{figure}
Its basic idea relies on choosing a random subset from among all matches, solving a problem of reduced dataset and establishing how many points from the original set satisfy the equation. When enough inliers are found, points that do not satisfy the equation can be removed from further processing. The example 2 degree-of-freefom problem of sample fitting can be seen in Figure \ref{fig:RANSACFitting}. \\

When internal camera parameters K are known, the image points found can be calibrated and expressed in camera reference position system. Such calibrated points satisfy the following essential matrix E equation:
\begin{equation} \label{eq:essentialEquation}
\textbf{x}_{c}^{'T} \textbf{E} \textbf{x}_{c} = 0
\end{equation} 
which is very similar to a fundamental equation \ref{eq:fundamntalEquation}. This results in:
\begin{equation} \label{eq:essentialFundamentalRelation}
\textbf{E} = \textbf{K}^{T} \textbf{F} \textbf{K}
\end{equation} 
with \textbf{K} being matrix representing the internal camera parameters. 

The most known methods for Fundamental and Essential matrix calculations are:
\begin{enumerate}
\item{\textbf{8-point algorithm}} - 8 corresponding points pairs in image must be found and all of them are used in order to resolve 8DOF matrix problem and find fundamental matrix. More informations can be found in \cite{8Point}.
\item{\textbf{5-point algorithm}} - for calibrated cameras only 5 points pairs have to be used to resolve 5DOF matrix problem and find proper essential matrix. More informations can be found in \cite{iterative5point} or \cite{Batra5point}.
\end{enumerate}
Using Equation \ref{eq:essentialFundamentalRelation} one can go from Fundamental to Essential matrix and back. Next section will explain, how essential matrix is used in order to find relative translation and rotation between cameras.
\subsection{Essential matrix decomposition}
In section 9.6.1 of Multiple View Geometry in Computer Vision (\cite{HartleyMultipleView}) it is shown in detail, how essential matrix can be decomposed using Singular Value Decomposition (SVD) to relative camera rotation \textbf{R} and translation \textbf{t}. In general SVD relay on decomposition one matrix to three other matrices, which multiplied give the initially decomposed matrix and where middle one is diagonal matrix. \\ 
Unfortunately, there are four possible solutions for such decomposition of essential matrix and it is not always possible to identify the correct one, especially when there is lot of unsuccessfully removed outliers present in corresponding points set. Figure \ref{fig:fourAmbig} presents the four possible situations of decomposition \textbf{E} to \textbf{R} and \textbf{T}. 
\begin{figure}[h!]
    \centering
    \includegraphics[width=0.8\textwidth]{fourAmbig}
    \caption[The four possible decompositions of \textbf{E}]{The four possible decompositions of \textbf{E}. Between the left and right sides there is a baseline reversal. Between the top and bottom rows camera B rotates 180$^{\circ}$ about baseline. Picture from page 260 of \cite{HartleyMultipleView}.}
    \label{fig:fourAmbig}
\end{figure}
\\
However it is good to note, that only in (a) is the reconstructed point in front of both cameras. Usually to determine proper solution it is enough using each of corresponding feature point to calculate its 3D positions and perform simple test checking if it lies in front of both cameras image planes. It may seem simple task, but from computer perspective it's hard to determine proper decomposition, when not all of outliers were removed during feature matching phase. Often outliers pair will show itself in front of both cameras for not correct solution.
This is the place in reconstruction pipeline where additional sensor data can be used to determine proper solution. In fact Sensor Fusion data can limit E matrix decomposition only to two solutions - (a) and (b) cases from Figure \ref{fig:fourAmbig}). What is most important that even in presence of many outliers it will produce proper perceptible models (cases (a) and (b) are baseline reversal). This will be explained in detail, when discussing enhancements to 8 and 5 point algorithms.
\subsection{Points Triangulation} \label{sec:points_triangulation}
Once internal and external camera parameters are calculated, the triangulation can be performed in order to acquire an affine reconstruction model (Figure \ref{fig:3Dreconstruction}).
\begin{figure}[h]
    \centering
    \includegraphics[width=0.8\textwidth]{3Dreconstruction}
    \caption[3D reconstruction from 2 images]{3D reconstruction from 2 images. Two rays crossing cameras centres and corresponding images points also cross in position of 3D point. [Slide from Photogrammetric Computer Vision from 2013/2014 winter semester at TU Berlin]}
    \label{fig:3Dreconstruction}
\end{figure}
Using \textbf{R} and \textbf{T} acquired from \textbf{E} decomposition and \textbf{K} from camera calibration, relation between 2D and 3D points can be expressed with following perspective projection matrices:
\begin{equation}
 \textbf{P}_{1} = \textbf{K} \begin{bmatrix}\textbf{I}\mid \textbf{0}\end{bmatrix} - \text{for first image,}
\end{equation}
\begin{equation}
 \textbf{P}_{2} = \textbf{K}  \begin{bmatrix}\textbf{R}\mid \textbf{t}\end{bmatrix} - \text{for second image,}
\end{equation}
where \textbf{I} is identity matrix, \textbf{0} is zero column vector, \textbf{R} is matrix representing relative rotation between cameras and \textbf{t} is relative translation vector between cameras. And now respectively:
\begin{equation}
 \textbf{x} = \textbf{P}_{1} \textbf{X} - \text{for first image,}
\end{equation}
\begin{equation}
 \textbf{x}^{'} = \textbf{P}_{2} \textbf{X} - \text{for second image,}
\end{equation}
where \textbf{X} is 3D position of point in space.
\\
Basically having relative cameras positions fully established (known internal parameters, rotation and translation) all rays connecting camera centres and corresponding image pairs will cross in place of their 3D location. Thus, it is good to note here that resolution and density of 3D models rely on picture resolution and number of matching correspondences found. 
The only element that cannot be determined in such a case is the scale, because only relative positions between cameras can be established. This process is described in more details in Chapter 10 of Hartley's book\cite{HartleyMultipleView}.
\section{Structure from Motion}
The term ''Structure from Motion (SfM)'' refers to the reconstruction performed from the consecutive sequences of a moving camera. It is a popular research topic and the two main approaches, namely the Pose Estimation and Homography Estimation, can be used for the purposes of reconstructing a 3D model of an object visible throughout images sequence.
Lecture of this section give solid ground for understanding what Chapter\ref{chap:structure_from_motion} is about.
\subsection{3D Pose Estimation}
Assuming that some of the 3D cloud points are already known, the matches between 2D features in a new image and 3D point cloud positions can be established. Such 3D-2D matches can be used to estimate the camera position. This allows for reconstructing new 3D points and merging them smoothly into a functional model. Unfortunately,  this process is also highly sensitive to the occurrence of outliers, therefore adequate measures have to be undertaken to reduce their influence. One of the main advantages of this method is its speed. On the other hand, its effectiveness relies strongly on the initially existing 3D cloud quality. 
\subsection{Structure Adjustment}
Special refinement methods can be used to compensate for an error of mismatched points propagating through images. They include. among others, Bundle Adjustment (BA). The algorithm, using the information concerning the corresponding matches between multiple sets of images, iteratively modifies either both the camera external parameters and 3D points positions or one of them\cite{website:bundle-adjustment}. The main disadvantage of the method is its execution time. It is too time-consuming to be used in real-time applications. The basic idea of BA is expressed in Figure \ref{fig:BundleAdjustment}.
\begin{figure}[h]
    \centering
    \includegraphics[width=0.8\textwidth]{BundleAdjustment}
    \caption[Bundle Adjustment process overview]{Bundle Adjustment process overview. Initially pictures around Pharaoh head from exactly the same distance on the circle. Due to presence of outliers, some of the estimated cameras were improperly estimated and  it could look like they were taken from different distances a). However after Bundle Adjustment process whole setup estimation was corrected to case b). Red circles indicates cameras, which positions were corrected in the process to the green ones. [Slide from Photogrammetric Computer Vision 2013/2014 winter semester at TU Berlin]}
    \label{fig:BundleAdjustment}
\end{figure}
Theoretically usage of Sensor Fusion initial rotation and translation estimation should produce close to actual values and thus result in better convergence speed during BA.
\section[Mobile Sensors overview]{Mobile Sensors overview\cite{website:androidSensorOverview}}
There are many sensors available in the nowadays smartphones, such as accelerometer, gyroscope, magnetometer, barometer, GPS, etc. All of them have their advantages and disadvantages, and thus the errors of some might be compensated with the strengths of the others in order to, for example, accurately compute camera rotation angles. Very good overview of sensors accuracy on example of Android can be found in Google Tech Talk called: "Sensor Fusion on Android Devices: A Revolution in Motion Processing" \cite{website:androidSensorFusion}.
\subsection[Accelerometer]{Accelerometer\cite{website:accelerometer}}
An accelerometer is a device that measures acceleration along 3 axes of the device. Generally, an accelerometer allows for measuring total acceleration by sensing what force is applied to its micro strings. An accelerometer, which lies still on a flat surface parallel to the Earth's surface will indicate approximately the value of earth's gravity acceleration (1G) - around 9.8 $\frac{m}{{s}^{2}}$. This gravitation vector can be used to calculate the relative camera rotation, but it is difficult to define to which direction the gravity vector points when the device is moving in not a linear manner and acceleration readings are the sum of gravity and movement acceleration. The gravity vector can be obtained from the accelerometer data, when the device is still and later tracked during unexpected movements with the usage of gyroscope sensor data. 
In case of Android accelerometers acceleration is returned in $\frac{m}{{s}^{2}}$.
\subsection[Gyroscope]{Gyroscope\cite{website:gyroscope}}
A gyroscope is a device for measuring or maintaining orientation, based on the principles of conservation of angular momentum. A standard gyroscope consists of a spinning wheel mounted on two gimbal rings, which allows it to rotate in all three axes. The spinning wheel will resist changes in orientation, due to an effect of the conservation of angular momentum. A conventional gyroscope measures orientation, in contrast to MEMS (Micro Electro-Mechanical System) types, which measure angular rate, and are therefore called rate-gyros. MEMS gyroscopes contain vibrating elements to measure the Coriolis effect. In the end the angular velocity can be calculated in each axis. It is important to note that whereas the accelerometer and the magnetometer measure acceleration and angle relative to the Earth, gyroscope measures angular velocity relative to the body.
In case of Android rate of rotation is returned in $\frac{rad}{s}$.
\subsection[Sensor Fusion]{Sensor Fusion\cite{website:sensorFustion}}
Sensor fusion is the process of combining sensory data derived from disparate sources in a way that the resulting information is to some extent more valuable than it would be possible should these sources be used individually. The term ''more valuable'' in this case may mean ''more accurate'', ''more complete'' or ''more dependable'', or refer to the result of an emerging view, such as stereoscopic vision (calculation of depth information by combining two-dimensional images from two cameras at slightly different viewpoints). 
In Android API available Sensor Fusion is made by combining accelerometer and gyroscope data. It allows on tracking gravity vector, which can be obtained when the device is not moving through rapid movements thanks to gyroscopic reading. The gravity vector can be decomposed in order to estimate the camera's rotation angles. In Figure \ref{fig:angles_from_gravity} reader can see overview of how gravity vector can be used to calculate rotation angles of the device. 
\begin{figure}[h!]
    \centering
    \includegraphics[width=0.8\textwidth]{angles_from_gravity}
    \caption[Getting rotation angles from gravity vector]{Getting rotation angles from gravity vector. In situation a) when device is not moving it is possible to determine the reference gravity vector values along each device axis. Later for situation b) to d) it is possible to calculate angles between current and reference gravity vector values on the corresponding device axes. Image from \cite{website:gravity_orientation}}
    \label{fig:angles_from_gravity}
\end{figure}
Unfortunately as shown for instance in \cite{sensorFusionSmartphones} and \cite{website:extendedSensorFusion} the calculated rotation angles may be few degrees drifted even when using Sensor Fusion. 
\\
Subtracting gravity from the actual acceleration measurements allows for estimating linear acceleration (Figure \ref{fig:linear_acceleration}). 
\begin{figure}[h!]
    \centering
    \includegraphics[width=0.8\textwidth]{linear_acceleration}
    \caption[Difference in readings of Android's acceleration and linear acceleration]{Difference in readings of Android's acceleration and linear acceleration. Having gravity vector pointing down (brown) and device accelerating in the right direction, acceleration vector (yellow) will show its minus influence on axis - x. For left situation when gravity vector is not known, final device acceleration readings will indicate orange value. However once gravity vector is known, its influence can be subtracted from acceleration reading, leaving pure device linear acceleration reading - right situation.}
    \label{fig:linear_acceleration}
\end{figure}
The use of linear acceleration makes it possible to measure the relative change of the device's translation. From fundamental physics distance traveled during acceleration movement moment is expressed by the following equation:
\begin{equation} \label{eq:trans_from_accel}
\textbf{s}_{\delta t} = \textbf{V}_{t} \cdot \delta t + \frac{\textbf{a} \cdot {\delta t}^{2}}{2},
\end{equation}
where $\textbf{V}_{t}$ represents current velocity of the device, $\delta t$ representing time difference, in which acceleration has happened and $\textbf{a}$ representing current linear acceleration vector. During acceleration velocity of the device changes according to following equation: 
\begin{equation}\label{eq:velo_accel}
\begin{array}{c}
\textbf{V}_{\delta t} = \textbf{a} \cdot \delta t \\
\textbf{V}_{t + \delta t} = \textbf{V}_{t} + \textbf{V}_{\delta t},
\end{array}
\end{equation}
However, as reader can see to get translation it is required to integrate acceleration over time twice, which can result in largely increasing the translation's estimation errors, when noisy data are available. 
As mentioned already in this section, \cite{website:androidSensorFusion} presents in particular how noisy mobile sensors prove to be in reality, especially when used during movement, thus translation estimation can be highly inaccurate.
Unfortunately there is no good official documentation, where Android linear acceleration reading error was calculated, but one of comprehensive study on Android Sensor Fusion and distance calculation using linear acceleration can be found in \cite{indoorPosition}. Also Chapter \ref{chap:recon_sensors} includes author's experiments on rotation and translation estimation using Sensor Fusion data.
% ---------------------------------------------------------------------------
% ----------------------- end of thesis sub-document ------------------------
% ---------------------------------------------------------------------------
						% aims of the project
% this file is called up by thesis.tex
% content in this file will be fed into the main document

%: ----------------------- name of chapter  -------------------------
\chapter{Related Work} % top level followed by section, subsection
There are many approaches to the reconstruction problems, starting with the raw one-by-one pixel analysis for dense reconstruction, through high level abstraction of objects recognition and extraction and ending with the light and shadows-based reconstruction \cite{objectRecAndLoc}\cite{combinedMonoSlam}.
However this thesis does not focus on high-level abstraction reconstruction, but on refining relative poses estimation steps. As described in the theoretical part, in order to estimate the relative positions of two cameras, an essential matrix must be decomposed. 
Since the basic epipolar geometry equation discovery, many scientist have introduced various ways to solve this linear problems, which differed primarily in terms of the accuracy and speed. One of the first was the 8-point algorithm \cite{8Point}, which can be used to compute a fundamental matrix. This can be done without any prior knowledge of the scene or camera external and internal parameters. Fundamental matrix has to be converted to essential matrix in order for the relative camera positions to be found and that is why it is necessary to calculate internal camera parameters. Next, the 5-point algorithm approaches were introduced \cite{Batra5point} \cite{Nister5point}. They used internal camera parameters for the proper essential matrix estimation. In his paper M. David Nister showed that the 5-point algorithm outperforms almost all similar algorithms in terms of accuracy and speed. Only the 8-point algorithm can be regarded as equally efficient. One of the state-of-art real-time and robust approaches is the iterative 5-point algorithm created by Vincent Lui and Tom Drummond\cite{iterative5point}.

Most of the algorithms are very sensitive to the occurrence of outliers. One of the most common approaches to use to tackle this problem is RANSAC modelling \cite{RansacRefine}.

A research group from Technishe Universität Berlin made an interesting comparison and evaluation of the methods discovered by 2008 \cite{HellwichEvaluation}. It was discovered that estimation of camera rotation is much more stable than translation calculation, which means that reconstruction effects are more sensitive to translation estimation errors than rotation matrix estimation. Furthermore, there are a lot of ambiguities regarding the choice of the correct solution of epipolar geometry equation, which cannot be easily resolved especially, when lot of outliers are present in matched features set. 

In certain situations, where external camera parameters as rotation and translation can be measured, more accurate algorithms were proposed.  In 2011 D. Scaramuzza from Zurich proposed a 1-point algorithm \cite{1point}, which shows how to describe and use a model of a camera mounted on a car to enhance 3D reconstruction. The 4-point algorithm introduced in 2013 using information of rotation angle in certain axis from additional sensor, as shown in paper \cite{4point}, can outperform even the 5-point algorithms. Lately the scientists have been creating more complex models to estimate relative stereometry, e.g. a team from Zurich proposed a way to enhance reconstruction with additional 6DOF sensor \cite{robustCameraImu}.

There are also non standard approaches like \cite{lineBasedPose}, where it is shown how to estimate the relative pose from three lines with two of the lines being parallel and orthogonal to the third one. Very accurate estimations can also be obtained for special camera model cases, when there is no rotation\cite{pureTransl}. All these references show that enhanced models often help to achieve more accurate or faster solutions to standard epipolar geometry problems.

Some of the researches drafted many scientific papers regarding rotation-improved solution space searching \cite{rotationSpaceSearch} \cite{Enqvist10stablestructure}. There are even some approaches involving the use of additional IMU accelerometer for the reconstruction process enhancement \cite{robustCameraImu}. This is what inspired author of this thesis to experiment with usage of Sensor Fusion data to enhance 3D reconstruction techniques.
% ---------------------------------------------------------------------------
%: ----------------------- end of thesis sub-document ------------------------
% ---------------------------------------------------------------------------
			
% this file is called up by thesis.tex
% content in this file will be fed into the main document

%: ----------------------- name of chapter  -------------------------
\chapter{Concept} % top level followed by section, subsection
As indicated in similar research it's very attractive to use additional data to enhance reconstruction and reduce ambiguity in finding correct solution of 3D reconstruction. Additional camera or model information help to implement faster, more stable and robust algorithms. This thesis will show how prior knowledge of rotation or translation acquired via mobile sensor fusion can be used to enhance process of 3D reconstruction from series images. When it comes to relaying on hand-held smartphones, collected sensor data are very noisy. This thesis, how even noisy informations can be used in struction reconstraction processes. Initially only camera rotation estimation was supposed to be used in the scope of this mater thesis. However first trials of reconstruction proved to be not sufficient enough and additional rotation error matrix estimations were proposed.
From analysis of theory and related works it seems that both epipolar equations and pose estimation techniques can be improved with additional rotation and translation information data.  Author of this thesis proposes environment, where user can decide what type of stargy to use.
\section{Requirements}
Proposed methodology needs as the input series of images with additional information about position of the camera - euclidan rotation and optionally translation. Usage of smartphone is actually not necessary. Any camera with SensorFusioned accelerometer and gyroscope(magnotometer is actually optional and as discussed in [TODO] has its advantages and disadvantages) capable of rotation and translation estimation can be used. Internal camera parameters need to be calculated before reconstruction process is started. Additional sensor data doesn't have to be very accurate. Noisy external camera parameters stil can be successfully used in order to enhance reconstruction process.
\section{Enhancing epipolar geometry equations with initial rotation matrix}
Initial pair reconstruction step is very important and needs to be accurate to let other images calculate relative position on basis of initially reconstructed 3D cloud points.
Taking standard fundamental geometry equation and relative camera based system ($P = \begin{bmatrix}I |0\end{bmatrix}, P' = \begin{bmatrix}R|t\end{bmatrix}$) we can note that:
\begin{equation} \label{eq:relativeFundamntal}
{x}_{'}^{T} * K^{-T} * \begin{bmatrix}T\end{bmatrix}_{x} * R * K^{-1} * x = 0
\end{equation}
It is can also be written that:
\begin{equation} \label{eq:skewTranslation}
\begin{bmatrix}T\end{bmatrix}_{x} = 
\begin{bmatrix*}[c]
 0 & -t_{z} & t_{y}\\
 t_{z} & 0 & -t_{x}\\
-t_{y} & t_{x} & 0 
\end{bmatrix*} 
where T = \begin{bmatrix}t_{x},t_{y},t_{z}\end{bmatrix}
\end{equation}
As we were discussing rotation can be distorted with noise. This can be written as:
\begin{equation} \label{eq:Rerror}
R = R_{error} * R_{init} 
\end{equation}
where $R_{init}$ is initial rotation matrix constructed from measured angles and $R_{error}$ is rotation error matrix.
Looking at this from different point of view \ref{eq:Rerror} can be interpreted as multipling two rotations matrix: 
One estimated, but close to local optimum initial rotation matrix and second one, which is responsible for correction of noise error. 
Basic idea of algorithm proposed in this thesis is instead of relative whole rotation matrix calculation, which can be acquired from Essential matrix SVD decomposition, only rotation $R_{error}$ can be estimated. In the end \ref{eq:relativeFundamntal} can be rewritten in form:
\begin{equation} \label{eq:relativeFundamntalEnhanced}
{x}_{'}^{T} * K^{-T} * \begin{bmatrix}T\end{bmatrix}_{x} * R_{error} * R_{init} * K^{-1} * x = 0
\end{equation}
Having:
\begin{equation} \label{eq:leftRelative}
\begin{array}{lcl}
h_{'}^{T} &=& {x}_{'}^{T} * K^{-T} \\
h &=& R_{init} * K^{-1} * x \\
G &=& \begin{bmatrix}T\end{bmatrix}_{x} * R_{error} \\
\end{array}
\end{equation}
With such notation one can notice that:
\begin{equation} \label{eq:alternativeEnhancedEquation}
{h}_{'}^{T} * G * h = 0
\end{equation}
which resembles already known fundamental(\ref:{eq:fundamntalEquation}) and essential equations (\ref{eq:essentialEquation}). Of cource $h_{'}$ and h both are expressed in homogenous coordinates in such situation. From analysis it is known that G has 6DOF: 3 due to unknown translation and another 3 due to unknown correction angles(created by rotation error matrx decomposition). From theory such matrix can be resolved for instance by both 5 and 8-point algorithms. So basicly standard fundamental and essential equation solvers can be used in order to retrieve both $\begin{bmatrix}T\end{bmatrix}_{x}$ and $R_{error}$.
In the end estimated $R_{error}$ and calculated $R_{init}$ has to be multiplied to retrieve new rotation estimation of R (\ref{eq:Rerror}).
From Appendix 6 "Multiple View Geometry in Computer Vision"(A6.9.1 \cite{HartleyMultipleView}) using Rodigues parametrization, when angles are small(and noise in initial rotation matrices estimations can be expresed by small angles) rotation matrix and thus $R_{error}$ as well is more or less equals to:
\begin{equation} \label{eq:rodiguesError}
R_{error} \cong 
\begin{bmatrix*}[c]
    1   &  -w_{z}&  w_{y}\\ 
 w_{z}  &    1   & -w_{x}\\
-w_{y}  &  w_{x} &   1
\end{bmatrix*}
\end{equation} 
Such criterium with special matrix design can be used when decomposing G to resolve some ambiguity in choosing proper solution. From standard 4 solution ambiguity with 2 possible rotations and translations, it can bre reduced to 2 possible translation calculations. Described concept is basis of implemented enhanced 8-point and 5-point algorithms

\subsection{Alternative 3-point algorithm for translation finding}
Starting \ref{eq:alternativeEnhancedEquation} it can be written that:
\begin{equation} \label{eq:alternative3point}
{x}_{'}^{T} * K^{-T} * \begin{bmatrix*}[c]
 0 & -t_{z} & t_{y}\\
 t_{z} & 0 & -t_{x}\\
-t_{y} & t_{x} & 0 
\end{bmatrix*} * R * K^{-1} * x = 0
\end{equation}
Having:
\begin{equation} \label{eq:leftRelative}
\begin{array}{lcl}
h_{'}^{T} &=& {x}_{'}^{T} * K^{-T} \\
h &=& K^{-1} * x \\
\end{array}
\end{equation}
and 
\begin{equation} \label{eq:alternative3point}
\begin{bmatrix*}[c]
h_{'1} & h_{'2} & h_{'3}
\end{bmatrix*}
* \begin{bmatrix*}[c]
 0 & -t_{z} & t_{y}\\
 t_{z} & 0 & -t_{x}\\
-t_{y} & t_{x} & 0 
\end{bmatrix*} 
* \begin{bmatrix*}[c]
h_{1} \\
h_{2} \\
h_{3}
\end{bmatrix*}
= 0
\end{equation}
and multiplying it we will end up with
\begin{equation} \label{eq:alternative3point}
h_{1}*h_{'2}*t_{z} - h_{1}*h_{'3}*t_{y} - h_{2}*h_{'1}*t_{z} + h_{2}*h_{'3}*t_{x} + h_{3}*h_{'1}*t_{y} - h_{3}*h_{'2}*t_{x}
= 0
\end{equation}
what can be grouped:
\begin{equation}
t_{x} * (h_{2}*h_{'3} - h_{3}*h_{'2}) + t_{y} * (h_{3}*h_{'1} - h_{1}*h_{'3}) + t_{z} * (h_{1}*h_{'2} - h_{2}*h_{'1}) = 0
\end{equation}
rewritten as:
\begin{equation} \label{eq:translation3point}
\begin{bmatrix*}[c]
t_{x} &
t_{y} &
t_{z}
\end{bmatrix*} * \begin{bmatrix*}[c]
(h_{2}*h_{'3} - h_{3}*h_{'2}) \\ 
(h_{3}*h_{'1} - h_{1}*h_{'3}) \\
(h_{1}*h_{'2} - h_{2}*h_{'1}) 
\end{bmatrix*} 
= 0
\end{equation}
and solved for instance with SVD with only 3 points. This is very fast way of translation estimation in reconstracted images. However in such situation overall accuracy strictly depends on precise measuerments of camera orientation.
\section{Pose estimation}
One of the main goal was to make more accurate and faster version of reconstruction algorithm. That is way relative pose estimation techniques were used to enrich models with additional points. In general, this approach produces less outliers than homography merging techniques. This is also allows to keep scale among reconstructed images. For any point in image, which has corresponding 3D point following condition is kept:
\begin{equation} \label{eq:projectionEquation}
 x = P * X
\end{equation}
where x is image point expressed in homogenous coordinates (x , y , 1) and X homogenous 3D  point (X , Y , Z , 1). 
Projection matrix of the camera design looks as follow: 
\begin{equation} \label{eq:projectionEquation}
 P = K * \begin{bmatrix}R\mid t\end{bmatrix}
\end{equation}
\subsection{Rotation enhancements}
Using similar thinking to initial pair reconstruction enhancements it can be noted that:
\begin{equation} \label{eq:projectionRotError1}
\begin{array}{rcl}
 x & = & K * \begin{bmatrix}Rinit + dR\mid t\end{bmatrix} * X \\
 x & = & K * \begin{bmatrix}Rinit\mid 0\end{bmatrix} * X + K * \begin{bmatrix}dR\mid t\end{bmatrix} * X \\
 x - K * \begin{bmatrix}Rinit\mid 0\end{bmatrix} & = & K * \begin{bmatrix}dR\mid t\end{bmatrix} * X
\end{array}
\end{equation}
Substituting $x_{m} = x - K * \begin{bmatrix}Rinit\mid 0\end{bmatrix}$ we get: 
\begin{equation} \label{eq:projectionRotError2}
x_{m} = K * \begin{bmatrix}dR\mid t\end{bmatrix} * X
\end{equation}
This can be solved using normal PNP calculating algorithms. Using rotation as initial solution for pose estimation can be used in order to focus only on rotation error and translation estimation.
\subsection{Rotation \& translation enhancements}
Similar to \ref{eq:projectionRotError1}:
\begin{equation} \label{eq:projectionRotError3}
\begin{array}{rcl}
 x & = & K * \begin{bmatrix}Rinit + dR\mid Tinit + dt\end{bmatrix} * X \\
 x & = & K * \begin{bmatrix}Rinit\mid Tinit\end{bmatrix} * X + K * \begin{bmatrix}dR\mid dt\end{bmatrix} * X \\
 x - K * \begin{bmatrix}Rinit\mid Tinit\end{bmatrix} & = & K * \begin{bmatrix}dR\mid dt\end{bmatrix} * X
\end{array}
\end{equation}
end in the end by Substituting $x_{n} = x - K * \begin{bmatrix}Rinit\mid Tinit\end{bmatrix}$ we get: 
\begin{equation} \label{eq:projectionRotError4}
x_{n} = K * \begin{bmatrix}dR\mid dt\end{bmatrix} * X
\end{equation}
This situation can also be solved using normal PNP[TODO reference] calculation algorithm. Using rotation and translation as initial solutions for pose estimation can be used in order to focus only on rotation error and translation error estimation.
\section{Known rotations \& translations}
In situation were accurate rotations and translations of cameras are known no aditional pose calculations are needed. Such situation is interesting, because everything needed for corresponding points triangulation is already known. However in mobile sensor data, such measurements are very noisy. However Bundle Adjustment can be used in order to refine measured calculated camera positions and final 3D cloud reconstruction.
\section{Reconstruction process strategy}
Finally all methods described in this section can be combined in different reconstruction strategies. In terms of initialization of 3D cloud point can be made using: 
%TODO zrobić lepszy styl
\begin{enumerate} 
\item \textbf{Known rotatations and translations}, which theoritcally allows on up-to-metrical reconstruction
\item \textbf{Known rotations}, where translation is estimated with alternative 3-point algorithm
\item \textbf{Noisy rotations}, where enhanced 8-point fundamental or 5-point essential algorithms are used in order to calculate rotation error and relative translation 
\item \textbf{Unknown external camera parameters}, where standard 8-point fundamental or 5-point essential algorithms are used in order to calculate Rotation and relative translation
\end{enumerate}
In terms of Pose Estimation following methodologies can be used:
\begin{enumerate}
\item \textbf{Known rotatation and translations}, where no additional calculations are required and up-to-metrical model can be acquired
\item \textbf{Initial rotation}, where relative translation needs to be calculated
\item \textbf{Nosy rotation and translation}, where both rotation and translation error needs to be calculated
\item \textbf{Unknown external camera parameters}, where standard pose estimation has to be used
\end{enumerate}

% ---------------------------------------------------------------------------
%: ----------------------- end of thesis sub-document ------------------------
% ---------------------------------------------------------------------------


% this file is called up by thesis.tex
% content in this file will be fed into the main document

%: ----------------------- name of chapter  -------------------------
\chapter{Implementation} % top level followed by section, subsection


%: ----------------------- contents from here ------------------------

\section{Environment}
\section{Project Structure}
\subsection{Android application}
\subsection{OSX CMake base project}
\section{Important Implementation Aspects}
\subsection{Custom Sensor Data File format}
\section{Graphical User Interface}
\section{Documentation}





% ---------------------------------------------------------------------------
%: ----------------------- end of thesis sub-document ------------------------
% ---------------------------------------------------------------------------


% this file is called up by thesis.tex
% content in this file will be fed into the main document

% change according to folder and file names
\ifpdf
    \graphicspath{{figures/}{figures/comparisons}}
\else
    \graphicspath{{figures/}{figures/comparisons}}
\fi


\chapter{Evaluation} % top level followed by section, subsection
All implemented algorithms were tested in terms of speed, accuracy and effectiveness. As regards the speed test, it included the measurement of the reconstruction execution time. In the accuracy testing Sampson Error measurement was used (this variable measures the distance between all points and their corresponding epipolar lines in an image). Effectiveness was tested with the use of the visual comparison of reconstructed 3D cloud points.
% ----------------------- contents from here ------------------------
\section{Acquiring datasets}
For the purpose of analysis the following datasets were captured with the implemented "Sensor Enhanced Images Camera" application and Nexus 5 camera:
\begin{enumerate}
\item Warsaw University of technology main building(4 images 1024x768pixels)
\item Advertisment Pole ??? 
\item Warsaw Buisness School Gate and Entrance ??
\item Warsaw Shopping Center Back???
\item Warsaw Shopping Center Front???
\end{enumerate}
Since most of the algorithms proposed in the (TODOreference to Concept) section require internal camera parametrs to be known, the camera used in the course of conducting this study was calibrated and its parameters were stored in $"out_camera_data.yml"$ file. All of these datasets can be found on attached CD or in Github repository.
\section{Test Environment}
All tests were perfomed on MacBook Air with 1.7GHz dual-core Intel Core i7 processor and 8GB 1600MHz DDR3 RAM using "Enhanced 3D Reconstructer" implemented as described in the Implementation chapter.  Numerical tests which allowed for measuring total errors and execution time were run on the "Warsaw University of Technology" dataset. Initial pair reconstruction ability of each method proposed was measured as well as various reconstruction strategies.  Finally, for each dataset the most effective methods were used to reconstruct sparse models. 
\section{Testing initial pair reconstrucion methods} \label{sec:Testin2Views}
The key question to be answered at this point was whether the proposed sensor enhancement gave better results than the standard algorithms.
The following methods were tested: TODO methods should be already described in Concept and implementation
\begin{enumerate}
\item \textbf{Standard 8-point} - based on [] the fundamental matrix decomposition, implemented in OpenCV
\item \textbf{Enhanced 8-point} - the proposed camera rotation enhanced version of the above 8-point algorithm
\item \textbf{Alternative 3-point} - the proposed 3-point algorithm for translation estimation.
\item \textbf{Known rotations and translations} - calculation from known cameras rotations and translations
\item \textbf{Standard essential 5-point} - based on [] the  essential matrix decomposition, implemented in []
\item \textbf{Enhanced essential 5-point} - similar to 2. rotation-enhanced version of the above 5-point algorithm
\end{enumerate}
\subsection{Accuracy - Epipolar lines correspondence}
In the case of initial pair images one of the most imporant factors include the epipolar constraint. With the use of a properly estimated fundamental matrix, drawing corresponding epipolar lines in both images is possible. Furthermore, points lying on two matching epipolars lines in different images can be easily matched. In other words, epipolar lines cross exactly the same points in both images. The more accurate they are the more corresponding pairs can be found, e.g. for the purposes of performing dense reconstruction. Sampson Error is one of the metrics that can be used in order to estimate the accuracy of epipolars lines; the smaller the error's value the more accurate the lines. 
\begin{figure}[b!]
  \begin{center}
    \begin{tikzpicture}
      \begin{axis}[
          width=\linewidth, % Scale the plot to \linewidth
          grid=major, % Display a grid
          grid style={dashed,gray!30}, % Set the style
          xtick = {100,500,1000,5000},
          xlabel=Sift features number, % Set the labels
          xmode=log,
          log ticks with fixed point,
          ymode=log,
          log ticks with fixed point,
          ylabel=Total Sampson Error(pixels),
          legend style={at={(0.5,-0.05)},anchor=north,cells={anchor=west}}, % Put the legend below the plot
        ]
        \addplot[mark=*,blue] table[x=Number of Sift features,y=8-point OpenCV,col sep=comma] {SampsonTotal.csv}; 
        \addplot[mark=*,red] table[x=Number of Sift features,y=8-point enhanced,col sep=comma] {SampsonTotal.csv};
        \addplot[mark=*,orange] table[x=Number of Sift features,y=Alternative 3-point,col sep=comma] {SampsonTotal.csv}; 
        \addplot[mark=*,pink] table[x=Number of Sift features,y=Known rot and trans,col sep=comma] {SampsonTotal.csv};
        \addplot[mark=*,green] table[x=Number of Sift features,y=Essential 5-point,col sep=comma] {SampsonTotal.csv};
        \addplot[mark=*,yellow] table[x=Number of Sift features,y=5-point enhanced,col sep=comma] {SampsonTotal.csv};
        \legend{Standard 8-point,Enhanced 8-point,Alternative 3-point,Known rotations and translations,Standard essential 5-point,Enhanced essential 5-point}
      \end{axis}
    \end{tikzpicture}
    \caption{Chart showing the sum of Sampson Errors in a picture(1024x768pixels) per various initial Sift features sets sizes}
    \label{plot:TotalSampsonError}
  \end{center}
\end{figure}
Its primary function is to measure the total distance between all points and their corresponding epipolar lines. However, each of the proposed methods has different outliers removal capabilities, therefore in order to compare their efficiency, the average of Samson Error per a corresponding pair was calculated.
The following chart \ref{plot:TotalSampsonError} shows the total Sampson Errors for different sets of initial SIFT features. The major observation made based on these results is that most of the algorithms remove outliers properly. As could have been expected the only one that is not efficient in this regard is the method involving drawing epipolar lines from heuristically estimated movement and noisy rotation, which produces significantly larger error than the other algorithms.
\begin{figure}[hb!]
  \begin{center}
    \begin{tikzpicture}
      \begin{axis}[
          width=\linewidth, % Scale the plot to \linewidth
          grid=major, % Display a grid
          grid style={dashed,gray!30}, % Set the style
          xtick = {100,500,1000,5000},
          xlabel=Sift features number, % Set the labels
          xmode=log,
          log ticks with fixed point,
          ymode=log,
          log ticks with fixed point,
          ylabel=Total Sampson Error per point(pixels),
          legend style={at={(0.5,-0.05)},anchor=north,cells={anchor=west}}, % Put the legend below the plot
        ]
        \addplot[mark=*,blue] table[x=Number of Sift features,y=8-point OpenCV,col sep=comma] {SampsonPerPoint.csv}; 
        \addplot[mark=*,red] table[x=Number of Sift features,y=8-point enhanced,col sep=comma] {SampsonPerPoint.csv};
        \addplot[mark=*,orange] table[x=Number of Sift features,y=Alternative 3-point,col sep=comma] {SampsonPerPoint.csv}; 
        \addplot[mark=*,pink] table[x=Number of Sift features,y=Known rot and trans,col sep=comma] {SampsonPerPoint.csv};
        \addplot[mark=*,green] table[x=Number of Sift features,y=Essential 5-point,col sep=comma] {SampsonPerPoint.csv};
        \addplot[mark=*,yellow] table[x=Number of Sift features,y=5-point enhanced,col sep=comma] {SampsonPerPoint.csv};
        \legend{Standard 8-point,Enhanced 8-point,Alternative 3-point,Known rotations and translations,Standard essential 5-point,Enhanced essential 5-point}
      \end{axis}
    \end{tikzpicture}
    \caption{Chart showing per point Sampson Error in a picture(1024x768pixels) per various initial Sift features sets sizes}
    \label{plot:SampsonErrorPerPoint}
  \end{center}
\end{figure}
Analysing the per pair Sampson Error results \ref{plot:SampsonErrorPerPoint} it can be noted that the proposed sensor enhanced methods did not improve either the 8-point or 5-point algorithms, but the 3-point algorithm developed for the purposes of this thesis proved to be much faster than the 8-point algorithm and also quite accurate. To present the discussed errors, estimated epipolar lines for 300 initial SIFT features set were drawn on the figures \ref{fig:SummaryEpiLines1300} - \ref{fig:SummaryEpiLines2300}.
\begin{figure}[b!]
    \centering
    \includegraphics[width=0.8\textwidth]{summary1Sift300}
    \caption{The results of drawing estimated epipolar lines for the Warsaw Univeristy Dataset with 300 Sift points. 1) Standard Fundamental 8-point algorithm (upper pair), 2) Rotation Enhanced Fundamental 8-point algorithm (middle pair), 3) Alternative 3-point algorithm (bottom pair) }
    \label{fig:SummaryEpiLines1300}
\end{figure}
\begin{figure}[ht!]
    \centering
    \includegraphics[width=0.8\textwidth]{summary2Sift300}
    \caption{The results of drawing estimated epipolar lines for the Warsaw Univeristy Dataset with 300 Sift points. 1) Fundamental matrix created from rotation and translation (upper pair), 2) Standard Essential matrix 5-point algorithm (middle pair), 3) Rotation Enhanced Essential 5-point algorithm (bottom pair) }
    \label{fig:SummaryEpiLines2300}
\end{figure}
It can be seen that both the standard 8-point algorithm and the proposed rotation-enhanced version return very good results in terms of epipolar lines accuracy. The alternative 3-point algorithm gives slightly worse results due to the uncompansated mobile sensors' noise. In this particular dataset the essential 5-point method was inefficient in producing proper essential matrix estimation, contrary to the proposed sensor enhanced 5-point algorithm, which found satisfactory correspondency in epipolar lines. \newline
To verify whether the epipolar lines calculation is influenced by the number of SIFT features, the same process was applied to 1000 SIFT features set(\ref{fig:SummaryEpiLines11000} - \ref{fig:SummaryEpiLines21000}). 
\begin{figure}[b!]
    \centering
    \includegraphics[width=0.8\textwidth]{summary1Sift1000}
    \caption{The results of drawing estimated epipolar lines for the Warsaw Univeristy Dataset with 1000 Sift points. 1) Standard Fundamental 8-point algorithm (upper pair), 2) Rotation Enhanced Fundamental 8-point algorithm (middle pair), 3) Alternative 3-point algorithm (bottom pair) }
    \label{fig:SummaryEpiLines11000}
\end{figure}
\begin{figure}[ht!]
    \centering
    \includegraphics[width=0.8\textwidth]{summary2Sift1000}
    \caption{The results of drawing estimated epipolar lines for the Warsaw Univeristy Dataset with 1000 Sift points. 1) Fundamental matrix created from rotation and translation (upper pair), 2) Standard Essential matrix 5-point algorithm (middle pair), 3)Rotation Enhanced Essential 5-point algorithm (bottom pair) }
    \label{fig:SummaryEpiLines21000}
\end{figure}
In general, the proposed enhanced algorithms are not better than standard versions in terms of accuracy of epipolar lines estimation. It is mostly because during calculation some of the noise in sensor data is propagated on epipolar geometry estimation. However, the enhanced versions are always close to optimal solutions, while the standard versions can often fail in terms of fundamental matrix estimation.
\clearpage

\subsection{Time comparison}
The chart \ref{plot:ExecutionTime} below shows the avaraged  execution time for 100 estimation attempts.
\begin{figure}[hb!]
  \begin{center}
    \begin{tikzpicture}
      \begin{axis}[
          width=\linewidth, % Scale the plot to \linewidth
          grid=major, % Display a grid
          grid style={dashed,gray!30}, % Set the style
          xtick = {100,500,1000,5000},
          xlabel=Sift features number, % Set the labels
          xmode=log,
          log ticks with fixed point,
          ymode=log,
          ylabel=Execution time(ms),
          legend style={at={(0.5,-0.05)},anchor=north,cells={anchor=west}}, % Put the legend below the plot
        ]
        \addplot[mark=*,blue] table[x=Number of Sift features,y=8-point OpenCV,col sep=comma] {ExecutionTime.csv}; 
        \addplot[mark=*,red] table[x=Number of Sift features,y=8-point enhanced,col sep=comma] {ExecutionTime.csv};
        \addplot[mark=*,orange] table[x=Number of Sift features,y=Alternative 3-point,col sep=comma] {ExecutionTime.csv}; 
        \addplot[mark=*,pink] table[x=Number of Sift features,y=Known rot and trans,col sep=comma] {ExecutionTime.csv};
        \addplot[mark=*,green] table[x=Number of Sift features,y=Essential 5-point,col sep=comma] {ExecutionTime.csv};
        \addplot[mark=*,yellow] table[x=Number of Sift features,y=5-point enhanced,col sep=comma] {ExecutionTime.csv};
        \legend{Standard 8-point algorithm,Enhanced 8-point algorithm,Alternative 3-point algorithm,Known rotations and translations,Standard essential 5-point algorithm,Enhanced essential 5-point algorithm}
      \end{axis}
    \end{tikzpicture}
    \caption{Execution time of the proposed algorithms(1024x768pixels) per initial SIFT feature set size}
    \label{plot:ExecutionTime}
  \end{center}
\end{figure}
Execution time of the 8-point algorithms are very similar, which is understandable, since their implementation is nearly identical and they differ only in the characteristics of the analysed dataset. It can be seen, however, that execution time values of the 5-point epipolar estimations differ significantly depending on the version. In this situation either finding optimal solution with initial rotation is much more difficult or implementation of these methods is significantly different in terms of memory allocation. Execution time of the 3-point algorithm is few times faster than the one of the 8-point algorithms. The reconstruction with the sole use of the known rotations and translations has a hundred times shorter execution time than the standard algorithms, but at the same time it does not produce properly correlated epipolar lines.

\section{Testing reconstruction strategies}
As was shown in \ref{sec:Testin2Views} not every initial reconstruction method gives good results. Only the most effective techniques were used to prepare a number of completely different strategies:
\begin{enumerate}
\item \textnormal{Standard 8-point + OpenCV Pose Estimation} 
\item \textnormal{Enhanced 8-point + OpenCV Pose Estimation} 
\item \textnormal{Enhanced 8-point + Initial Rotation and Translation OpenCV Pose Estimation} 
\item \textnormal{Alternative 3-point + OpenCV Pose Estimation}
\item \textnormal{Alternative 3-point + Initial Rotation and Translation OpenCV Pose Estimation}
%\item \textnormal{Known rotations and translations} 
\end{enumerate}
The first method gives the basic reference to standard 3D reconstruction strategy. The second one was chosen for its consistency in 3D reconstraction. It has never failed to produce solution close to optimal. The third one was prepared in order to determine how the enhanced pose estimation influences the final reconstruction outcomes. The strategies number four and five were used to see if the models can be produced faster without impacting the final accuracy too heavily. Finally, the sixth method was proposed in order to verify wheter the entire reconstruction process can be performed using the sensor data only.

\subsection{Accuracy}
Accuracy of the 3D reconstruction was measured using Bundle Adjustment algorithm from SBA library[ref]. It allows for calculating the error based on the projective constraint both before and after Bundle Adjustment.
The tests performed with different strategies  for the "Warsaw Univeristy" dataset are presented on \ref{plot:BAError}. The significant differences in the initial error can be explained by different numbers of reconstructed points after initial phase and inconsistent and unknown scale of the finally reconstructed models. The enhanced initial pair reconstruction and pose estimation methods have bigger impact on Bundle Adjustment error reduction.
\begin{figure}[ht!]
  \begin{center}
    \begin{tikzpicture}
      \begin{axis}[
          width=\linewidth, % Scale the plot to \linewidth
          grid=major, % Display a grid
          grid style={dashed,gray!30}, % Set the style
          xtick = {1,2,3,4,5},
          xticklabels = {Std8 -point+NormalPose,Enhanced8-point+NormalPose,Enhanced8-point+EnhancedPose, Alternative3-point+NormalPose,Alternative3-point+EnhancedPose},          
          xlabel=Reconstruction strategy, % Set the labels
          log ticks with fixed point,
          ymode=log,
          ylabel=Reconstruction error(in model units),
          x tick label style={rotate=90,anchor=east},
          legend style={at={(0.5,1.05)},anchor=south,cells={anchor=west}}, % Put the legend below the plot
        ]
        \addplot[mark=*,blue] table[x=ReconMethods,y=Error before BA,col sep=comma] {BAerror2.csv}; 
        \addplot[mark=*,red] table[x=ReconMethods,y=Error after BA,col sep=comma] {BAerror2.csv};
        \legend{Reconstruction Error before Bundle Adjustment,Reconstruction Error after Bundle Adjustment}
      \end{axis}
    \end{tikzpicture}
    \caption{Influence of Bundle Adjustment on the models produced with different reconstruction strategies }
    \label{plot:BAError}
  \end{center}
\end{figure}
\clearpage
In general, Bundle Adjustmet process allows to rearrange and modify 3D points positions and change external parameters of estimated cameras. However, in the case of the enhanced algorithms the estimated camera positions are already very close to their optimal orientations. This allows Bundle Adjustment method to focus mainly on 3D points modifications. It can be observed that the enhanced pose estimation results in further reduction of BA error when compared to the standard strategies. 

\subsection{Execution time}
The chart \ref{plot:ReconstructionWithoutBA} presentes the execution time without Bundle Adjustment. Comparing it to the execution times in \ref{plot:ExecutionTime} one can notice that most of the time is allocated for SIFT correspondences matching, which constitutes a bottleneck in the proposed reconstruction methodology [reference to concept pose esitmation choosing]. Furthermore, the reconstruction time was also measured with Bundle Adjustment process at the end(\ref{plot:ReconstructionWithBA}). It was found that Bundle Adjustment works significantly better with enhanced initial pair reconstruction and pose estimation. In that case the cloud points are better organised than in the standard reconstruction, which results in faster convergence of BA. Sample difference between the cloud points before and after BA can be seen in \ref{fig:BundleAdjustmentComparison}
\begin{figure}[b!]
    \centering
    \includegraphics[width=\textwidth]{bundleAdjustmentComparison}
    \caption{3D point clouds before Bundle Adjustment (upper) and after (bottom) for enhanced 8-point with enhanced pose estimation. Warsaw University of Technology dataset with 1000 SIFT corresponding features (left - front, right - side)}
    \label{fig:BundleAdjustmentComparison}
\end{figure}
\begin{figure}[h!]
  \begin{center}
    \begin{tikzpicture}
      \begin{axis}[
          width=\linewidth, % Scale the plot to \linewidth
          grid=major, % Display a grid
          grid style={dashed,gray!30}, % Set the style
          xtick = {100,500,1000,5000},
          xlabel=Sift features number, % Set the labels
          xmode=log,
          log ticks with fixed point,
          ymode=normal,
          ylabel=Execution time(ms),
          legend style={at={(0.5,-0.05)},anchor=north,cells={anchor=west}}, % Put the legend below the plot
        ]
        \addplot[mark=*,blue] table[x=Number of Sift features,y=8-pointInit+NormalPoseEstim,col sep=comma] {ReconstructTotal.csv}; 
        \addplot[mark=*,red] table[x=Number of Sift features,y=Enhanced8-point+NormalPoseEstim,col sep=comma] {ReconstructTotal.csv};
        \addplot[mark=*,orange] table[x=Number of Sift features,y=Enhanced8-point+EnhancedPoseEstim,col sep=comma] {ReconstructTotal.csv}; 
        \addplot[mark=*,pink] table[x=Number of Sift features,y=Alternative3-point+NormalPoseEstim,col sep=comma] {ReconstructTotal.csv};
        \addplot[mark=*,green] table[x=Number of Sift features,y=Alternative3-point+EnhancedPoseEstim,col sep=comma] {ReconstructTotal.csv};
%        \addplot[mark=*,yellow] table[x=Number of Sift features,y=Known Rotations and Translations,col sep=comma] {ReconstructTotal.csv};
        \legend{Standard 8-point initialization + Normal OpenCV Pose Estimation,Enhanced 8-point initialization + Normal Pose Estimation,Enhanced 8-point initiazliation + Enhanced Pose Estim, Alternative 3-point initiazliation + Normal Pose Estimation, Alternative 3-point initialization + Enhanced Pose Estimation}
      \end{axis}
    \end{tikzpicture}
    \caption{Total reconstruction execution time (4 images with resolution 1024x768pixels) per SIFT features set size}
    \label{plot:ReconstructionWithoutBA}
  \end{center}
\end{figure}

%TODO At this point an interesting observation was made that pose estimation time is relatively short as for correspondence matching. Similarily triangulation also proved to be fast.\newline
\begin{figure}[h!]
  \begin{center}
    \begin{tikzpicture}
      \begin{axis}[
          width=\linewidth, % Scale the plot to \linewidth
          grid=major, % Display a grid
          grid style={dashed,gray!30}, % Set the style
          xtick = {100,500,1000,5000},
          xlabel=Sift features number, % Set the labels
          xmode=log,
          log ticks with fixed point,
          ymode=normal,
          ylabel=Execution time(ms),
          scaled ticks = false,
          xticklabel style={/pgf/number format/.cd,fixed,precision=5},
          legend style={at={(0.5,-0.05)},anchor=north,cells={anchor=west}}, % Put the legend below the plot
        ]
        \addplot[mark=*,blue] table[x=Number of Sift features,y=8-pointInit+NormalPoseEstim,col sep=comma] {ReconstructBA.csv}; 
        \addplot[mark=*,red] table[x=Number of Sift features,y=Enhanced8-point+NormalPoseEstim,col sep=comma] {ReconstructBA.csv};
        \addplot[mark=*,orange] table[x=Number of Sift features,y=Enhanced8-point+EnhancedPoseEstim,col sep=comma] {ReconstructBA.csv}; 
        \addplot[mark=*,pink] table[x=Number of Sift features,y=Alternative3-point+NormalPoseEstim,col sep=comma] {ReconstructBA.csv};
        \addplot[mark=*,green] table[x=Number of Sift features,y=Alternative3-point+EnhancedPoseEstim,col sep=comma] {ReconstructBA.csv};
%        \addplot[mark=*,yellow] table[x=Number of Sift features,y=Known Rotations and Translations,col sep=comma] {ReconstructTotal.csv};
        \legend{Standard 8-point initialization + Normal OpenCV Pose Estimation,Enhanced 8-point initialization + Normal Pose Estimation,Enhanced 8-point initiazliation + Enhanced Pose Estim, Alternative 3-point initiazliation + Normal Pose Estimation, Alternative 3-point initialization + Enhanced Pose Estimation}
      \end{axis}
    \end{tikzpicture}
    \caption{Total execution time of reconstruction with Bundle Adjustment (4 images with resolution 1024x768pixels) per SIFT features set size}
    \label{plot:ReconstructionWithBA}
  \end{center}
\end{figure}
\clearpage

\section{Effectivness}
The numerical measuerments used do not always correspond to the proper 3D model reconstructions. The following pages present the reconstruction effects of the proposed strategies. Figure \ref{fig:uni4000Comparison} shows the reconstructed effects for different initialization pair methods and 4000 SIFT features set. It can be observed that both standard and enhaced 8-point methods produces very good results, which differ only in terms of final reconstruction scale. Alternative 3-point algorithm produces sligthly worse and distorted models due to uncompenstaed noise in the rotations of the camera used. A model reconstructed from the known rotations and translations is very distorted, but nonetheless stil recognizable. It could prove useful should a very fast reconstruction be needed.
\newline
The matter is slightly different when 400 SIFT feature points are used. [reference] In the first case all algorithms were able to find solutions close to optimal. However, in the second attempt the traingulation test, which is used to identify proper deconmposition in standard 8-point approach, failed and produced an unrecognizable model (\ref{fig:FailCaseFundamental}).
\newline
It can be seen that pose estimations enhancments result primarily in the reduction of outliers (\ref{fig:PoseEstimationMethodComparison}).
\newline
Figure \ref{fig:UniNone4000} shows that reconstruction from the known rotations and translations produces many outliers.
\newline
More reconstructed models can be found in [TODO Materials]
\begin{figure}[p]
    \centering
    \includegraphics[width=0.9\textwidth]{uni4000Comparison}
    \caption{Reconstructed models for the proposed initial reconstruction methods and 4000 SIFT features. From upper left to bottom right: 1) standard 8-point, 2) enhanced 8-point, 3) alternative 3-point, 4) known rotations and translations, 5) standard 5-point, 6) enhanced 5-point}
    \label{fig:uni4000Comparison}
\end{figure}
\begin{figure}[t!]
    \centering
    \includegraphics[width=0.9\textwidth]{uni400Comparison}
    \caption{Reconstructed models for the proposed initial reconstruction methods and 400 SIFT features. From upper left to bottom right: 1) standard 8-point, 2) enhanced 8-point, 3) alternative 3-point, 4) known rotations and translations, 5) standard 5-point, 6) enhanced 5-point}
    \label{fig:uni400Comparison}
\end{figure}
\begin{figure}[p]
    \centering
    \includegraphics[width=0.9\textwidth]{FailCaseFundamental}
    \caption{Fail test case of Standard 8-point triangulation(left) in comparison to fortunate reconstruction(right)}
    \label{fig:FailCaseFundamental}
\end{figure}
\begin{figure}[p]
    \centering
    \includegraphics[width=0.9\textwidth]{PoseEstimationMethodComparison}
    \caption{Pose estimation methods comparison (Views from front and side). Left: Normal Pose Estimation, right: Enhanced Rotation and Translation Pose Estimation}. Less outliers appear in the reconstruction if enhancment is applied.
    \label{fig:PoseEstimationMethodComparison}
\end{figure}
\begin{figure}[p]
    \centering
    \includegraphics[height=18cm]{uniNone4000}
    \caption{Reconstruction results from known translations and rotations from different angles. The upper one shows the front face of building, the others present views from side angles. In the reconstructed model many outliers are present.}
    \label{fig:UniNone4000}
\end{figure}

% ---------------------------------------------------------------------------
% ----------------------- end of thesis sub-document ------------------------
% ---------------------------------------------------------------------------

% this file is called up by thesis.tex
% content in this file will be fed into the main document

\chapter{Conclusion} % top level followed by section, subsection


% ----------------------- contents from here ------------------------

\section{Summary}
\section{Dissemination}
Who uses your component or who will use it? Industry projects, EU projects, open 
source...? Is it integrated into a larger environment? Did you publish any papers? 
\section{Problems Encountered}
\section{Future work}




% ---------------------------------------------------------------------------
% ----------------------- end of thesis sub-document ------------------------
% ---------------------------------------------------------------------------              
\ifpdf
    \graphicspath{{figures/}{figures/comparisons}}
\else
    \graphicspath{{figures/}{figures/comparisons}}
\fi
% this file is called up by thesis.tex
% content in this file will be fed into the main document

\chapter{Additional materials} % top level followed by section, subsection
\begin{lstlisting}[language=json,firstnumber=1, float, label={lst:json_file} caption=Sample ''sensor.txt'' file from ''Sensor Enhanced Images Camera'' application]
[
   {
      "photoPath":"20141210_145643/0.jpg",
      "rotationMatrix":[],
      "azimuth":121.88075,
      "posX":-1.7521392107009888,
      "posY":-1.4345977306365967,
      "posZ":0.9248641133308411,
      "photoId":1,
      "pitch":13.867888,
      "roll":178.16968
   },
   {
      "photoPath":"20141210_145643/1.jpg",
      "rotationMatrix":[],
      ],
      "azimuth":110.66925,
      "posX":-4.244707942008972,
      "posY":-1.1443554759025574,
      "posZ":0.9647054933011532,
      "photoId":2,
      "pitch":11.625216,
      "roll":179.73383
   }
   ...
]
\end{lstlisting} %TODO opis i caption z referencją do implementation
\clearpage

\begin{table}[p]
\centering
\begin{adjustbox}{width=1\linewidth}
\begin{tabular}{l|l|l|l|l}
\textbf{100 SIFT Features}   & \textbf{Total Sampson Error} & \textbf{Sampson Error per Point} & \textbf{Points left} & \textbf{Execution time(ms)} \\ \hline
\textbf{8-point OpenCV}      & 20.5793             & 0.478588                & 43          & 0.4387             \\ \hline
\textbf{Alternative 3-point} & 112.749             & 4.17588                 & 27          & 0.1484             \\ \hline
\textbf{8-point enhanced}    & 67.2559             & 1.56409                 & 43          & 0.3867             \\ \hline
\textbf{Known rot and trans} & 3501.23             & 83.3625                 & 42          & 0.0001             \\ \hline
\textbf{Essential 5-point}   & 43.5866             & 1.06309                 & 41          & 0.684              \\ \hline
\textbf{5-point enhanced}    & 1863.66             & 45.4552                 & 41          & 13.4643            \\
\end{tabular}
\end{adjustbox}
\caption[Efficiency table of proposed methods for 100 SIFT features in Warsaw University of technology dataset]{Efficiency table of proposed methods for 100 SIFT features in Warsaw University of technology dataset. Columns: Total Sampson Error, Average Sampson error per point, Amount of points left after outliers removal, Execution time}
\label{table:Efficiency100Sift}
\end{table}

\begin{table}[p]
\centering
\begin{adjustbox}{width=1\linewidth}
\begin{tabular}{l|l|l|l|l}
\textbf{500 SIFT features}   & \textbf{Total Sampson Error} & \textbf{Sampson Error per Point} & \textbf{Points left} \& \textbf{Execution time(ms)} \\ \hline
\textbf{8-point OpenCV}      & 100.584             & 0.543697                & 185         & 1.0833             \\ \hline
\textbf{Alternative 3-point} & 220.722             & 2.53704                 & 87          & 0.2692             \\ \hline
\textbf{8-point enhanced}    & 112.7               & 0.609189                & 185         & 0.8362             \\ \hline
\textbf{Known rot and trans} & 14770.7             & 80.2756                 & 184         & 0.0001             \\ \hline
\textbf{Essential 5-point}   & 404.098             & 2.29601                 & 176         & 3.8827             \\ \hline
\textbf{5-point enhanced}    & 501.987             & 2.8522                  & 176         & 47.0683            \\
\end{tabular}
\end{adjustbox}
\caption[Efficiency table of proposed methods for 500 SIFT features in Warsaw University of technology dataset]{Efficiency table of proposed methods for 500 SIFT features in Warsaw University of technology dataset. Columns: Total Sampson Error, Average Sampson error per point, Amount of points left after outliers removal, Execution time}
\label{table:Efficiency500Sift}
\end{table}

\begin{table}[p]
\centering
\begin{adjustbox}{width=1\linewidth}
\begin{tabular}{l|l|l|l|l}
\textbf{1000 SIFT features}  & \textbf{Total Sampson Error} & \textbf{Sampson Error per Point} & \textbf{Points left} & \textbf{Execution time(ms)} \\ \hline
\textbf{8-point OpenCV}      & 265.637             & 0.781287                & 340         & 1.5055             \\ \hline
\textbf{Alternative 3-point} & 640.895             & 4.1083                  & 156         & 0.4725             \\ \hline
\textbf{8-point enhanced}    & 295.152             & 0.868093                & 340         & 1.8099             \\ \hline
\textbf{Known rot and trans} & 27394.4             & 77.1673                 & 355         & 0.0001             \\ \hline
\textbf{Essential 5-point}   & 518.293             & 1.85768                 & 279         & 5.4482             \\ \hline
\textbf{5-point enhanced}    & 349.393             & 1.2523                  & 279         & 204.2998           \\
\end{tabular}
\end{adjustbox}
\caption[Efficiency table of proposed methods for 1000 SIFT features in Warsaw University of technology dataset]{Efficiency table of proposed methods for 1000 SIFT features in Warsaw University of technology dataset. Columns: Total Sampson Error, Average Sampson error per point, Amount of points left after outliers removal, Execution time}
\label{table:Efficiency1000Sift}
\end{table}

\begin{table}[p]
\centering
\begin{adjustbox}{width=1\linewidth}
\begin{tabular}{l|l|l|l|l}
\textbf{5000 SIFT features}      & \textbf{Total Sampson Error} & \textbf{Sampson Error per Point} & \textbf{Points left} & \textbf{Execution time(ms)} \\ \hline
\textbf{8-point OpenCV}          & 517.189             & 0.439413                & 1177        & 2.187              \\ \hline
\textbf{Alternative 3-point}     & 1879.98             & 3.28094                 & 573         & 0.8286             \\ \hline
\textbf{8-point enhanced}        & 1087.78             & 0.924199                & 1177        & 5.4395             \\ \hline
\textbf{Known rot and trans}     & 93951.1             & 77.9677                 & 1205        & 0.0001             \\ \hline
\textbf{Essential 5-point}       & 1949.53             & 1.95736                 & 996         & 15.2223            \\ \hline
\textbf{5-point enhanced}        & 19966.6             & 20.0468                 & 996         & 355.464            \\ 
\end{tabular}
\end{adjustbox}
\caption[Efficiency table of proposed methods for 5000 SIFT features in Warsaw University of technology dataset.]{Efficiency table of proposed methods for 5000 SIFT features in Warsaw University of technology dataset. Columns: Total Sampson Error, Average Sampson error per point, Amount of points left after outliers removal, Execution time}
\label{table:Efficiency5000Sift}
\end{table}

\begin{figure}[p]
    \centering
    \includegraphics[height=18cm]{Gate4000Comparison}
    \caption[Reconstruction results of ''Gate'' dataset]{Reconstruction results of ''Gate'' dataset. Upper pair - standard OpenCV 8-point algorithm (left - front view, right - above view). Bottom pair - rotation enhanced 8-point algorithm.}
\end{figure}

\begin{figure}[p]
    \centering
    \includegraphics[height=18cm]{Pole4000Comparison}
    \caption[Reconstruction results of ''Pole'' dataset]{Reconstruction results of ''Pole'' dataset. Upper pair - standard OpenCV 8-point algorithm (left - front view, right - above view). Middle pair - rotation enhanced 8-point algorithm. Bottom pair - reconstruction from known rotations and translations.}
\end{figure}

\begin{figure}[p]
    \centering
    \includegraphics[height=18cm]{GalleryBack4000Comparison}
    \caption[Reconstruction results of ''Gallery Back'' dataset]{Reconstruction results of ''Gallery Back'' dataset. Upper pair - standard OpenCV 8-point algorithm (left - front view, right - above view). Bottom pair - rotation enhanced 8-point algorithm.}
\end{figure}

\begin{figure}[p]
    \centering
    \includegraphics[width=\linewidth]{GallerFrontReconstruction}
    \caption[Reconstruction results of ''Gallery Front'' dataset]{Reconstruction results of ''Gallery Front'' dataset using rotation enhanced 8-point algorithm (left - front view, right - above view).}
\end{figure}


% ---------------------------------------------------------------------------
%: ----------------------- end of thesis sub-document ------------------------
% ---------------------------------------------------------------------------



 






        % description of lab methods


% --------------------------------------------------------------
%:                  BACK MATTER: appendices, refs,..
% --------------------------------------------------------------

% the back matter: appendix and references close the thesis


%: ----------------------- bibliography ------------------------

% The section below defines how references are listed and formatted
% The default below is 2 columns, small font, complete author names.
% Entries are also linked back to the page number in the text and to external URL if provided in the BibTex file.

% PhDbiblio-url2 = names small caps, title bold & hyperlinked, link to page 
%\begin{multicols}{2} % \begin{multicols}{ # columns}[ header text][ space]
\begin{footnotesize} % tiny(5) < scriptsize(7) < footnotesize(8) < small (9)

\bibliographystyle{Latex/Classes/PhDbiblio-url2} % Title is link if provided
\renewcommand{\bibname}{References} % changes the header; default: Bibliography
\nocite{*}

\bibliography{9_backmatter/references.bib} % adjust this to fit your BibTex file

\end{footnotesize}
%\end{multicols}

% --------------------------------------------------------------
% Various bibliography styles exit. Replace above style as desired.

% in-text refs: (1) (1; 2)
% ref list: alphabetical; author(s) in small caps; initials last name; page(s)
%\bibliographystyle{Latex/Classes/PhDbiblio-case} % title forced lower case
%\bibliographystyle{Latex/Classes/PhDbiblio-bold} % title as in bibtex but bold
%\bibliographystyle{Latex/Classes/PhDbiblio-url} % bold + www link if provided

%\bibliographystyle{Latex/Classes/jmb} % calls style file jmb.bst
% in-text refs: author (year) without brackets
% ref list: alphabetical; author(s) in normal font; last name, initials; page(s)

%\bibliographystyle{plainnat} % calls style file plainnat.bst
% in-text refs: author (year) without brackets
% (this works with package natbib)


% --------------------------------------------------------------

% according to Dresden med fac summary has to be at the end
%
% Thesis Abstract -----------------------------------------------------


%\begin{abstractslong}    %uncommenting this line, gives a different abstract heading
\begin{abstracts}        %this creates the heading for the abstract page

The main subject of this thesis addresses the possibility of enhancing the existing 3D reconstruction algorithms with additional Android sensor fusion data. The primary fundamentals behind the reconstruction techniques and the problems related to them are explained in order to introduce the reader to these topics. Related works are discussed in order to explain the strengths and weaknesses of the state-of-art reconstruction approches. The proposed enhancements using initial rotation in the standard 8-point and 5-point algorithms are explained in detail. Both of them allow for focusing on correcting rotation matrix error instead of calculating the rotation itself. This, in turn, serves for reducing ambiguity of the decomposition of essential matrix to the relative rotation and translation of the cameras. The primary idea behind the implementation of the 3-point algorithm for translation is also described. The most important aspects of the implemented $"Sensor Enhanced Image Camera"$ Android application and $"Enhanced 3D Reconstructer"$ were discussed.
The three proposed initial pair reconstruction methods were evaluated, which showed that each algorithm is able to improve certain asspects of the 3D reconstruction. Different reconstruction strategies were evaluated. As a result it was discovered that the enhanced structure computation can be faster and more accurate than the standard methodology. Using initial rotation and translation pose estimation results in significantly faster convergence and better error reduction with Bundle Adjustment. This thesis covers only some notions of the Structure from Motion, therefore plans for future development were  established.

\end{abstracts}
\clearpage
\renewcommand{\abstractname}{Zusammenfassung}
\begin{abstracts}        %this creates the heading for the abstract page

Das Hauptthema dieser Arbeit befasst sich mit der M{\"o}glichkeit der Verbesserung der vorhandenen 3D-Rekonstruktionsalgorithmen mit zus{\"a}tzlichen Android Sensorfusion Daten. Die wichtigsten Grundlagen hinter den Rekonstruktionstechniken und die damit verbundenen Probleme sind, um den Leser zu diesen Themen vorstellen erl{\"a}utert. {\"A}hnliche Arbeiten sind erforderlich, um die Festigkeiten und Schw{\"a}chen der Stand-der-Technik Rekonstruktion Ansätze erkl{\"a}ren diskutiert. Die vorgeschlagenen Verbesserungen mit Anfangsrotation in den Standard 8-Punkte und 5-Punkte-Algorithmen werden im Detail erkl{\"a}rt. Beide von ihnen k{\"o}nnen zur Fokussierung auf die Rotationsmatrix Fehler Korrektur anstelle der Berechnung des Rotation selbst. Dies wiederum serviert zur Reduzierung der Mehrdeutigkeit des Essential-Matrix Zersetzungs zu der relativen Rotation und Translation den Kameras. Die Hauptidee hinter der Umsetzung der 3-Punkt-Algorithmus f{\"u}r die Translation ist ebenfalls beschrieben. Die wichtigsten Aspekte die implementierten $"Sensor Verbesserte Bildkamera "$ Android-Anwendung und $ "Verbesserte 3D Reconstructer "$  wurden diskutiert.
Die drei vorgeschlagenen ersten Paar Rekonstruktionsmethoden wurden bewertet, die zeigten, dass jeder Algorithmus kann bestimmte asspects der 3D-Rekonstruktion zu verbessern. Verschiedene Rekonstruktionsstrategien wurden bewertet. Als Ergebnis wurde festgestellt, dass die verbesserte Struktur Berechnung kann schneller und genauer als die Standardmethode sein. Mit  anfänglichen Rotation und Translation  Posenschätzung resultiert in signifikant schneller Konvergenz und besseres Fehlerreduktion mit B{\"u}ndelausgleichung. Diese Diplomarbeit umfasst nur einige Vorstellungen von der Struktur aus Bewegung, daher plant f{\"u}r die zuk{\"u}nftige Entwicklung entstanden.

\end{abstracts}
%\end{abstractlongs}


% ---------------------------------------------------------------------- 


%: Declaration of originality
\include{9_backmatter/declaration}



\end{document}
