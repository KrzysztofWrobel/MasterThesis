% this file is called up by thesis.tex
% content in this file will be fed into the main document

%: ----------------------- introduction file header -----------------------
\chapter{Introduction}
% the code below specifies where the figures are stored
\ifpdf
    \graphicspath{{1_introduction/figures/PNG/}{1_introduction/figures/PDF/}{1_introduction/figures/}}
\else
    \graphicspath{{1_introduction/figures/EPS/}{1_introduction/figures/}}
\fi

% ----------------------------------------------------------------------
%: ----------------------- introduction content ----------------------- 
% ----------------------------------------------------------------------

Mobile and wearable devices are becoming more and more popular. Modern smartphones despite having extremely good camera's also use advanced sensor's, like Accelerometers, Gyroscope, Magnetometer, Barometer etc.. There is also a big need and growing market of Augmented Reality (AR) and Virtual Reality(VR). That's why image analysis and recognition, as well as 3-D reconstruction techniques are really hot topic. Unfortunately algorithms that support these techniques are very time and memory consuming, that's why it's really hard to run them on mobile devices, which have many limitations in terms of CPU speed and RAM memory capacity.

Today many devices are capable of 3D reconstruction. One of them is very popular Kinect\cite{kinect}. It has ability of performing real-time 3D cloud point generation, but it has very high accuracy of reconstruction, but in the same time it's very expansive and not exactly mobile.

\section{Purpose of this thesis} % section headings are printed smaller than chapter names
Author of this document will present the reader with an overview of the idea of 3D reconstruction. This thesis also inculdes brief description of related research in this area. After short analysis of efficiency, accuracy and common problems of few chosen algorithms this thesis will propose their enhacment with data acquired with sensors, which can be found in smartphones. At the end author presents evaluation and discuss test results. TODO finish

\section{Scope}
The author researched, how Accelerometer, Gyroscope and Magnetometer can be used in order to improve Fundamental, Essential matrix and also relative Pose Estimation. Unfortunately raw data sensors are really noisy and it's really hard to use them individually to enhance reconstruction. However there is a way to combine these data together in order to compensate error of each individual sensor. The term describing this process is called "Sensor Fusion". This data fusion allows to estimates in real time a relative or global(in term of earth magnetic field) rotation and translation of the device. TODO finish
\section{Initial assumptions}
The general process of 3-D reconstruction is quite broad, that's why the author of the thesis focus only on certain aspects of this topic. That's why author didn't write algorithms from the scratch, but built his algorithms on top of OpenCV library and "Relative Pose Esitmation" Open-source project setuped by ....... . In terms of sensor fusion, currently state of art approach is used by most of big Mobile Operating Systems(Android, iOS, Windows Phone). That's why author used Sensor Fusion API from API and only wrote what's needed it terms of getting rotation and translation of the smartphone camera, when acquiring images for his research. TODO finish
\section{Thesis Outline}
In Chapter 2 something something and so on
\\*
In Chapter 3
\\*
In Chapter 4
\\*
In Chapter 5
\\*
In Chapter 6
\\*
In Chapter 7
\\*
In Chapter 8

% ----------------------------------------------------------------------



