% this file is called up by thesis.tex
% content in this file will be fed into the main document

%: ----------------------- introduction file header -----------------------
\chapter{Introduction}
% the code below specifies where the figures are stored
\ifpdf
    \graphicspath{{1_introduction/figures/PNG/}{1_introduction/figures/PDF/}{1_introduction/figures/}}
\else
    \graphicspath{{1_introduction/figures/EPS/}{1_introduction/figures/}}
\fi

% ----------------------------------------------------------------------
%: ----------------------- introduction content ----------------------- 
% ----------------------------------------------------------------------

Mobile and wearable devices are becoming more and more popular. Modern smartphones not only have extremely good cameras, but also use advanced sensors, like accelerometers, gyroscope, magnetometer, barometer etc. There is also a big need and growing market of Augmented Reality (AR) and Virtual Reality (VR) applications. That is why image analysis and recognition, as well as 3D reconstruction techniques are important directions in the development of the modern technology and computer science. Unfortunately, algorithms that support these techniques are very time and memory consuming and this makes them difficult to be run on mobile devices, which have many limitations in terms of CPU speed and RAM memory capacity.

\section{Purpose of this thesis} % section headings are printed smaller than chapter names
The author of this thesis will present the reader with an overview of the idea of 3D reconstruction techniques. This thesis also inculdes a brief description of related research conducted in this area. After short analysis of efficiency, accuracy and common problems of several algorithms selected, this thesis will propose their enhancement with the use of data acquired from Android sensor fusion. Two applications were created in order to verify and test the proposed algorithms and reconstruction strategies. Both Android and Desktop applications will be discussed as regards their essential implementation aspects. This study also explains the testing results. Towards the end, the conclusions, encountered errors and problems are discussed. Finally, future plans for the research development is presented.
\section{Scope}
The author researched how the fusion of accelerometer, gyroscope and magnetometer can be used in order to improve fundamental and essential matrix calculations as well as the relative pose estimation. Unfortunately, the raw data sensors are too noisy and therefore not reliable enough to rely solely on them in the course of enhancing the 3D reconstruction. Heuristic measurment of translation used for enhancment with linear acceleration was also conceptualised and implemented by author. In this research initial rotation enhanced 8-point and 5-point reconstruction algorithms are introduced. The alternative 3-point algorithm for translation estimation has also been implemented and is discussed herein.
This thesis covers only initial pair images reconstruction and relative pose estimation for Structure from Motion computation. The best methodology for corresponding feature computation is not sbject to research, but other possibilities of point matching are discussed. Finally, it is researched how the improved versions of the standard 8-point and 5-point algorithms influence convergence speed and error reduction of Bundle Adjustment.
\section{Initial assumptions}
The field of 3D reconstruction is quite broad, that is why the author of the thesis focuses primarily on certain aspects thereof. First of all it was not the author's intention to write all algorithms anew, but to built his versions based on the OpenCV library and opensource project called "Relative Pose Esitmation" created by TODO . In terms of sensor fusion, the current state-of-the-art approach can be found on Android platform. 3D reconstruction will be processed in a seperate desktop application.
\section{Thesis Outline}
In \textbf{Chapter 2} the fundamental technology behind 3D reconstruction and Structure from Motion is discussed. Also sensors used to perform sensor fusion on Android platform are briefly introduced.
\\*
\textbf{Chapter 3} contains a brief overview of related research. The discoveries of utmost importance, which lead to creating the concept of the proposed methods are described. 
\\*
In \textbf{Chapter 4} the theoretical concept of the proposed 8-point and 5-point rotation-enhanced algorithms is described. Furthermore, an alternative approach to the 3-point translation estimation is introduced. Finally, the author proposes a few most promising Structure from Motion reconstruction strategies.
\\*
\textbf{Chapter 5} discusses the most important implementation asspects of rotation and translation measurements. In additiona to that enhancements modifications to standard 8-point and 5-point algorithms are discussed. The 3-point translation estimation approach and relative pose ehancements implementation is also covered therein.
\\*
The objective of \textbf{Chapter 6} is to present and discuss the evaluation of the proposed initial pair reconstruction methods and strategies.
\\*
\textbf{Chapter 7} presents the author's conslusions observed duringin the course of investigating the subject and implementing ideas, as well as the encounterd problems. The author also outlined the further development plans for the researched solutions and approaches.

% ----------------------------------------------------------------------



