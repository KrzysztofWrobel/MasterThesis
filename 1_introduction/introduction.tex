% this file is called up by thesis.tex
% content in this file will be fed into the main document

%: ----------------------- introduction file header -----------------------
\chapter{Introduction}
% the code below specifies where the figures are stored
\ifpdf
    \graphicspath{{1_introduction/figures/PNG/}{1_introduction/figures/PDF/}{1_introduction/figures/}}
\else
    \graphicspath{{1_introduction/figures/EPS/}{1_introduction/figures/}}
\fi

% ----------------------------------------------------------------------
%: ----------------------- introduction content ----------------------- 
% ----------------------------------------------------------------------

Mobile and wearable devices are becoming more and more popular. Modern smartphones are not only equipped with first-rate cameras, but also use advanced sensors, like accelerometers, gyroscope, magnetometer, barometer etc. There is also a big need for Augmented Reality (AR) and Virtual Reality (VR) applications which is evidenced by the constantly growing market of such solutions. That is why image analysis and recognition, as well as 3D reconstruction techniques are important directions in the development of the modern technology and computer science. Unfortunately, algorithms that support these techniques are very time and memory consuming and this makes them difficult to be run on mobile devices, which have many limitations in terms of CPU speed and RAM memory capacity. Those algorithms can be improved with additional data, even if these data are noisy or have limited precision. 

\section{Purpose of this thesis} % section headings are printed smaller than chapter names
The author of this thesis will present the reader with an overview of the idea of 3D reconstruction techniques from multiple images and a description of related research conducted in this area. After analysis of efficiency, accuracy and common problems of reconstruction algorithms, the author will propose his own algorithm aimed at enhancing the process with the use of data acquired from Android Sensor Fusion. In particular, an approach of using such sensor data for initial camera pose estimation in order to resolve ambiguities of 3D reconstruction techniques will be presented. Such ambiguities are especially hard to resolve, when there is hard to find proper matching point correspondences in images. The author shows how standard algorithms which focus on finding relative rotation matrix can be modified to use Sensor Fusion data in order to estimate rotation matrix and focus only on finding "error-correcting" rotation matrix. To prove author's theoretical assumptions two applications were created: Android Application to capture images along with Sensor Fusion data and desktop C++ standalone application to perform test of proposed algorithms. Both Android and desktop applications will be discussed as regards their essential implementation aspects. This study also explains the testing results in the applied environment and setup. Towards the end, the conclusions, encountered errors and problems are discussed. Finally, future plans for the research development are presented.

TODO 
This thesis will show how the prior knowledge of rotation or translation acquired via mobile sensor fusion can be used to enhance process of 3D reconstruction from a series of images. When it comes to relying on hand-held smartphones, the collected sensor data are very noisy. This thesis shows how even noisy information can be used in the reconstruction processes.
\section{Scope}
The author researched how the sensor fusion of accelerometer, gyroscope and magnetometer data (in particular that one performed on Android platform) can be used in order to improve fundamental and essential matrix calculations as well as their decomposition to the relative pose estimation. In the beginning author had an idea to relay solely on rotation and translation estimation from sensor data, but unfortunately, the raw data sensors are too noisy and therefore not reliable enough to rely solely on them in the course of enhancing the 3D reconstruction \cite{website:androidSensorFusion}. Nevertheless these heuristic measurements of positions, especially translation between cameras were also conceptualised and implemented by author and are part of this thesis. The alternative 3-point algorithm for camera translation estimation has also been implemented by author and is discussed herein. Although this "3-point translation" algorithm used along with Sensor Fusion camera rotation estimation proved to be not accurate enough, it was important step of author's research and is described in this thesis. 
In this research Sensor Fusion data enhanced 8-point and 5-point reconstruction algorithms are introduced, which are the derivative of theoretical discussion on using additional Sensor Fusion data in spatial reconstruction techniques. It is also researched how the improved versions of the standard 8-point and 5-point algorithms influence convergence speed and error reduction of Bundle Adjustment, which is also a part of modern 3D reconstruction process. This thesis covers not only initial images pair reconstruction, but also relative pose estimation methods for Structure from Motion computation, when series of images are available. Unfortunately, the best methodology for images corresponding feature matching is not the subject of this research. This is also the reason, why author does not discuss, implement or present dense 3D reconstruction pipeline. Therefore all presented reconstructed models consist only from white supporting points and have no texture rendered on them.

\section{Initial assumptions}
The field of 3D reconstruction is quite broad, that is why the author of the thesis focuses primarily on certain aspects thereof. First of all it was not the author's intention to write all algorithms anew, but to built his versions based on the OpenCV library and open-source project called ''Relative Pose Estimation'' created by Bo Li \cite{website:relativePoseLibrary}. In terms of sensor fusion, the current state-of-the-art approach can be found and accessed in Android platform SDK. Due to limitations of compilation and debugging of C++ code on Android platform, 3D reconstruction will be processed in a separate desktop application. What is important that such approach does not influence value of this research. The main goal of this document is to research possibilities of enhancing standard algorithms used for 3D reconstruction with additional sensor data. Once this research is done all coding work regarding implementing new 3D reconstruction pipeline in C++ can be easily ported with help of Android Native Development Kit (NDK).  

\section{Thesis outline}
In \textbf{Chapter 2} the fundamental theory behind 3D reconstruction and Structure from Motion is discussed. Also sensors used to perform sensor fusion on Android platform are introduced and their strengths and weaknesses are described.
\\*
\textbf{Chapter 3} contains a overview of related research. The discoveries of utmost importance, which lead to creating the concept of the proposed methods are also described there. 
\\*
In \textbf{Chapter 4}
\\*
\textbf{Chapter 5}
\\*
TODO
\\*
TODO
\\*
TODO
\\*
%In \textbf{Chapter 4} the theoretical concept of the proposed 8-point and 5-point rotation-enhanced algorithms is described. Furthermore, an alternative approach of "3-point translation estimation", which was an author's initial idea for 3D reconstruction enhancement is introduced. Finally, the author proposes a few most promising Structure from Motion reconstruction strategies, which could be used for instance in hand-held camera applications written for Android platform.
%\\*
%\textbf{Chapter 5} discusses the most important implementation aspects of rotation and translation measurements and estimation. In addition to that enhancements and modifications to standard 8-point and 5-point algorithms are discussed. The 3-point translation estimation approach and relative pose enhancements implementation is also covered therein.
\\*
The objective of \textbf{Chapter 9} is to present and discuss the evaluation of the proposed initial pair reconstruction methods and different strategies of Structure From Motion computations.
\\*
\textbf{Chapter 10} presents the author's conclusions observed in the course of investigating the subject and implementing ideas, as well as the encountered problems. The author also outlined the further development plans for the researched solutions and approaches.

% ----------------------------------------------------------------------



