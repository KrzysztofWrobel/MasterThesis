% this file is called up by thesis.tex
% content in this file will be fed into the main document

% change according to folder and file names
\ifpdf
    \graphicspath{{figures/}{figures/comparisons}}
\else
    \graphicspath{{figures/}{figures/comparisons}}
\fi


\chapter{Evaluation} % top level followed by section, subsection
All implemented algorithms were tested in terms of speed, accuracy and effectiveness. As regards the speed test, it included the measurement of the reconstruction execution time. In the accuracy testing Sampson Error measurement was used (this variable measures the distance between all points and their corresponding epipolar lines in an image). Effectiveness was tested using visual comparison of reconstructed 3D cloud points.
% ----------------------- contents from here ------------------------
\section{Acquiring datasets}
For the purpose of analysis the following datasets were captured with implemented "Sensor Enhanced Images Camera" and Nexsus 5 camera:
\begin{enumerate}
\item Warsaw University of technology main building(4 images 1024x768pixels)
\item Advertisment Pole ??? 
\item Warsaw Buisness School Gate and Entrance ??
\item Warsaw Shopping Center Back???
\item Warsaw Shopping Center Front???
\end{enumerate}
Since most of the algorithms proposed in the (TODOreference to Concept) Concept section require internal camera parametrs to be known, the camera used in the course of conducting this study was calibrated and its parameters were stored in $"out_camera_data.yml"$ file. All of these datasets can be found in attached CD or Github repository.
\section{Test Environment}
All tests were perfomed on MacBook Air with 1.7GHz dual-core Intel Core i7 processor and 8GB 1600MHz DDR3 RAM using "Enhanced 3D Reconstructer" implemented as described in Implementation section.  Numerical tests which allowed for measuring total errors and execution time were run on "Warsaw University of Technology" dataset. Initial pair reconstruction ability of each method proposed was measured as well as various reconstruction strategies.  Finally, for each dataset the most effective methods were used to reconstruct sparse models. 
\section{Testing initial pair reconstrucion methods} \label{sec:Testin2Views}
The key question to be answered at this point was whether the proposed sensor enhancement gave better results than standard algorithms.
The following methods were tested: TODO methods should be already described in Concept and implementation
\begin{enumerate}
\item \textbf{Standard 8-point} - based on [] Fundamental matrix decomposition, implemented in OpenCV
\item \textbf{Enhanced 8-point} - proposed camera rotation enhanced version of above 8-point algorithm
\item \textbf{Alternative 3-point} - proposed 3-point algorithm to translation estimation.
\item \textbf{Known rotations and translations} - calculation from known cameras rotations and translations
\item \textbf{Standard essential 5-point} - based on [] Essential matrix decomposition, implemented in []
\item \textbf{Enhanced essential 5-point} - similar to 2. improvements rotation enhanced version of above 5-point algorithm
\end{enumerate}
\subsection{Accuracy - Epipolar lines correspondence}
In the case of initial pair images one of the most imporant factors include the epipolar constraint. Using a properly estimated fundamental matrix drawing corresponding epipolar lines in both images is possible. Furthermore, points lying on two matching epipolars lines in different images can be easily matched. In other words, epipolar lines cross exactly the same points in both images. The more accurate they are the more corresponding pairs can be found, e.g. for th epurposes of performing dense reconstruction. Sampson Error is one of metrics that can be used in order to estimate the accuracy of epipolars lines; the smaller the error's value the more accurate the lines. 
\begin{figure}[b!]
  \begin{center}
    \begin{tikzpicture}
      \begin{axis}[
          width=\linewidth, % Scale the plot to \linewidth
          grid=major, % Display a grid
          grid style={dashed,gray!30}, % Set the style
          xtick = {100,500,1000,5000},
          xlabel=Sift features number, % Set the labels
          xmode=log,
          log ticks with fixed point,
          ymode=log,
          log ticks with fixed point,
          ylabel=Total Sampson Error(pixels),
          legend style={at={(0.5,-0.05)},anchor=north,cells={anchor=west}}, % Put the legend below the plot
        ]
        \addplot[mark=*,blue] table[x=Number of Sift features,y=8-point OpenCV,col sep=comma] {SampsonTotal.csv}; 
        \addplot[mark=*,red] table[x=Number of Sift features,y=8-point enhanced,col sep=comma] {SampsonTotal.csv};
        \addplot[mark=*,orange] table[x=Number of Sift features,y=Alternative 3-point,col sep=comma] {SampsonTotal.csv}; 
        \addplot[mark=*,pink] table[x=Number of Sift features,y=Known rot and trans,col sep=comma] {SampsonTotal.csv};
        \addplot[mark=*,green] table[x=Number of Sift features,y=Essential 5-point,col sep=comma] {SampsonTotal.csv};
        \addplot[mark=*,yellow] table[x=Number of Sift features,y=5-point enhanced,col sep=comma] {SampsonTotal.csv};
        \legend{Standard 8-point,Enhanced 8-point,Alternative 3-point,Known rotations and translations,Standard essential 5-point,Enhanced essential 5-point}
      \end{axis}
    \end{tikzpicture}
    \caption{Chart showing the sum of Sampson Errors in a picture(1024x768pixels) per various initial SIFT features sets sizes}
    \label{plot:TotalSampsonError}
  \end{center}
\end{figure}
Its primary function is to measure the sum of distance between all points and their corresponding epipolar lines. However, each of proposed methods has different outliers removal capabilities, therefore in order to compare their efficiency the average of Samson Error per a corresponding pair was calculated.
The following chart \ref{plot:TotalSampsonError} shows the total Sampson Errors for different sets of initial SIFT features. The major observation made based on these results is that most of the algorithms remove outliers properly. As could have been expected the only one that is not efficient in this regard is the method involving drawing epipolar lines from heuristically estimated movement and noisy rotation, which produces significantly larger error than the other algorithms.
\begin{figure}[hb!]
  \begin{center}
    \begin{tikzpicture}
      \begin{axis}[
          width=\linewidth, % Scale the plot to \linewidth
          grid=major, % Display a grid
          grid style={dashed,gray!30}, % Set the style
          xtick = {100,500,1000,5000},
          xlabel=Sift features number, % Set the labels
          xmode=log,
          log ticks with fixed point,
          ymode=log,
          log ticks with fixed point,
          ylabel=Total Sampson Error per point(pixels),
          legend style={at={(0.5,-0.05)},anchor=north,cells={anchor=west}}, % Put the legend below the plot
        ]
        \addplot[mark=*,blue] table[x=Number of Sift features,y=8-point OpenCV,col sep=comma] {SampsonPerPoint.csv}; 
        \addplot[mark=*,red] table[x=Number of Sift features,y=8-point enhanced,col sep=comma] {SampsonPerPoint.csv};
        \addplot[mark=*,orange] table[x=Number of Sift features,y=Alternative 3-point,col sep=comma] {SampsonPerPoint.csv}; 
        \addplot[mark=*,pink] table[x=Number of Sift features,y=Known rot and trans,col sep=comma] {SampsonPerPoint.csv};
        \addplot[mark=*,green] table[x=Number of Sift features,y=Essential 5-point,col sep=comma] {SampsonPerPoint.csv};
        \addplot[mark=*,yellow] table[x=Number of Sift features,y=5-point enhanced,col sep=comma] {SampsonPerPoint.csv};
        \legend{Standard 8-point,Enhanced 8-point,Alternative 3-point,Known rotations and translations,Standard essential 5-point,Enhanced essential 5-point}
      \end{axis}
    \end{tikzpicture}
    \caption{Chart showing per point Sampson Error in a picture(1024x768pixels) per various initial SIFT features sets sizes}
    \label{plot:SampsonErrorPerPoint}
  \end{center}
\end{figure}
Analysing the Per Pair Sampson Error results \ref{plot:SampsonErrorPerPoint} it can be noted that the proposed sensor enhanced methods did not improve either the 8-point or 5-point algorithms, but the 3-point algorithm developed for the purposes of this thesis proved to be much faster than the 8-point algorithm and also quite accurate. To present the discussed errors, estimated epipolar lines for 300 initial SIFT features set were drawn on the figures \ref{fig:SummaryEpiLines1300} - \ref{fig:SummaryEpiLines2300}.
\begin{figure}[b!]
    \centering
    \includegraphics[width=0.8\textwidth]{summary1Sift300}
    \caption{The results of drawing estimated epipolar lines on Warsaw Univeristy Dataset with 300 SIFT points. 1) Standard Fundamental 8-point algorithm (upper pair), 2) Rotation Enhanced Fundamental 8-point algorithm (middle pair), 3) Alternative 3-point algorithm (bottom pair) }
    \label{fig:SummaryEpiLines1300}
\end{figure}
\begin{figure}[ht!]
    \centering
    \includegraphics[width=0.8\textwidth]{summary2Sift300}
    \caption{The results of drawing estimated epipolar lines on Warsaw Univeristy Dataset with 300 SIFT points. 1) Fundamental matrix created from rotation and translation (upper pair), 2) Standard Essential matrix 5-point algorithm (middle pair), 3) Rotation Enhanced Essential 5-point algorithm (bottom pair) }
    \label{fig:SummaryEpiLines2300}
\end{figure}
It can be seen that both standard 8-point algorithm and the proposed rotation enhanced version return very good results in terms of epipolar lines accuracy. The alternative 3-point algorithm gives slightly worse results due to uncompansated mobile sensors noise. In this particular dataset the essential 5-point method was inefficient in producing proper essential matrix estimation, contrary to the proposed sensor enhanced 5-point algorithm, which found satisfactory correspondency in epipolar lines. \newline
To verify whether the epipolar lines calculation is influenced by the number of SIFT features, the same process was applied to 1000 SIFT features set(\ref{fig:SummaryEpiLines11000} - \ref{fig:SummaryEpiLines21000}). 
\begin{figure}[b!]
    \centering
    \includegraphics[width=0.8\textwidth]{summary1Sift1000}
    \caption{Results of drawing estimated epipolar lines on Warsaw Univeristy Dataset with 1000 SIFT points. 1) Standard Fundamental 8-point algorithm (upper pair), 2) Rotation Enhanced Fundamental 8-point algorithm (middle pair), 3) Alternative 3-point algorithm (bottom pair) }
    \label{fig:SummaryEpiLines11000}
\end{figure}
\begin{figure}[ht!]
    \centering
    \includegraphics[width=0.8\textwidth]{summary2Sift1000}
    \caption{Results of drawing estimated epipolar lines on Warsaw Univeristy Dataset with 1000 SIFT points. 1) Fundamental matrix created from rotation and translation (upper pair), 2) Standard Essential matrix 5-point algorithm (middle pair), 3)Rotation Enhanced Essential 5-point algorithm (bottom pair) }
    \label{fig:SummaryEpiLines21000}
\end{figure}
In general, the proposed enhanced algorithms are not better than standard versions in terms of accuracy of epipolar lines estimation. Mostly because during calculation some of the noise in sensor data is propagated on Epipolar geometry estiamion. However enhanced versions are always close to optimal solutions. Standard versions for instance can fail sometimes in terms of Fundamental matrix estimation.
\clearpage

\subsection{Time comparison}
On chart \ref{plot:ExecutionTime} avaraged  execution time for 100 estimation attempts was presented.
\begin{figure}[hb!]
  \begin{center}
    \begin{tikzpicture}
      \begin{axis}[
          width=\linewidth, % Scale the plot to \linewidth
          grid=major, % Display a grid
          grid style={dashed,gray!30}, % Set the style
          xtick = {100,500,1000,5000},
          xlabel=Sift features number, % Set the labels
          xmode=log,
          log ticks with fixed point,
          ymode=log,
          ylabel=Execution time(ms),
          legend style={at={(0.5,-0.05)},anchor=north,cells={anchor=west}}, % Put the legend below the plot
        ]
        \addplot[mark=*,blue] table[x=Number of Sift features,y=8-point OpenCV,col sep=comma] {ExecutionTime.csv}; 
        \addplot[mark=*,red] table[x=Number of Sift features,y=8-point enhanced,col sep=comma] {ExecutionTime.csv};
        \addplot[mark=*,orange] table[x=Number of Sift features,y=Alternative 3-point,col sep=comma] {ExecutionTime.csv}; 
        \addplot[mark=*,pink] table[x=Number of Sift features,y=Known rot and trans,col sep=comma] {ExecutionTime.csv};
        \addplot[mark=*,green] table[x=Number of Sift features,y=Essential 5-point,col sep=comma] {ExecutionTime.csv};
        \addplot[mark=*,yellow] table[x=Number of Sift features,y=5-point enhanced,col sep=comma] {ExecutionTime.csv};
        \legend{Standard 8-point algorithm,Enhanced 8-point algorithm,Alternative 3-point algorithm,Known rotations and translations,Standard essential 5-point algorithm,Enhanced essential 5-point algorithm}
      \end{axis}
    \end{tikzpicture}
    \caption{Execution time of proposed algorithms(1024x768pixels) in comparison to initial SIFT feature set}
    \label{plot:ExecutionTime}
  \end{center}
\end{figure}
Execution time of 8-point versions are very similar, which is understandable, because they implementation is pretty much the same and they differ only characteristic of analysed dataset. It can be seen that execution time of 5-point epipolar estimations versions differ much. In this situation either finding optimal solution with initial rotation is much harder or implementation of these methods are very different in terms of memory allocation. Execution time of 3-points algorithm is few times faster than 8-point algorithms. Estimation using only known rotations and translations has few times smaller order execution time than standard algorithms, but as we know from previous section gives uncorreleted epipolar lines.

\section{Testing reconstruction strategies}
As was shown in \ref{sec:Testin2Views} not every initial reconstruction methods gives good results. Only the most effective techniques were used to prepare few completely different strategies:
\begin{enumerate}
\item \textnormal{Standard 8-point + OpenCV Pose Estimation} 
\item \textnormal{Enhanced 8-point + OpenCV Pose Estimation} 
\item \textnormal{Enhanced 8-point + Initial Rotation and Translation OpenCV Pose Estimation} 
\item \textnormal{Alternative 3-point + OpenCV Pose Estimation}
\item \textnormal{Alternative 3-point + Initial Rotation and Translation OpenCV Pose Estimation}
%\item \textnormal{Known rotations and translations} 
\end{enumerate}
First method gives us basic reference to standard 3D reconstruction strategy. Second one was choosen, because extremly consistence in 3D reconstraction. It has never failed to produce solution close to optimal. Thrid was prepared in order to check, how enhanced Pose Estimation influence final reconstruction. Forth and fifth strategies were used to see, if models can be produced faster without much impact on final accuracy. Finally sixth one was proposed in order to check, wheter whole reconstruction process can be performed with only sensor data.

\subsection{Accuracy}
Accuracy of 3D reconstraction was measured using Bundle Adjustment algorithm from SBA library[ref]. It allows to calculate error based on projective constratin both before and after Bundle Adjustment.
Performed with different strategis tests for "Warsaw Univeristy" dataset were plotted on \ref{plot:BAError}. Big differences in initial error can be explained different number of reconstructed points after initial phase and inconsistent unknown scale of finally reconstracted models. Enhanced initial pair reconstruction and pose estimation methods have bigger impact on Bundle Adjustment error reduction.
\begin{figure}[ht!]
  \begin{center}
    \begin{tikzpicture}
      \begin{axis}[
          width=\linewidth, % Scale the plot to \linewidth
          grid=major, % Display a grid
          grid style={dashed,gray!30}, % Set the style
          xtick = {1,2,3,4,5},
          xticklabels = {Std8 -point+NormalPose,Enhanced8-point+NormalPose,Enhanced8-point+EnhancedPose, Alternative3-point+NormalPose,Alternative3-point+EnhancedPose},          
          xlabel=Reconstruction strategy, % Set the labels
          log ticks with fixed point,
          ymode=log,
          ylabel=Reconstruction error(in model units),
          x tick label style={rotate=90,anchor=east},
          legend style={at={(0.5,1.05)},anchor=south,cells={anchor=west}}, % Put the legend below the plot
        ]
        \addplot[mark=*,blue] table[x=ReconMethods,y=Error before BA,col sep=comma] {BAerror2.csv}; 
        \addplot[mark=*,red] table[x=ReconMethods,y=Error after BA,col sep=comma] {BAerror2.csv};
        \legend{Reconstruction Error before Bundle Adjustment,Reconstruction Error after Bundle Adjustment}
      \end{axis}
    \end{tikzpicture}
    \caption{Influence of Bundle adjustment on models produced with different reconstruction strategies }
    \label{plot:BAError}
  \end{center}
\end{figure}
\clearpage
In general, Bundle Adjustmet process allows to rearrange and modify 3D points positions and also change external paramaters of estimated cameras. However, in case of enhanced algoritms estimated camera positions are already very close to their optimal estimations.  Such situation allows Bundle Adjustment method to focus mainly on 3D points modifications. It can be seen that enhanced Pose Estimation in comparison to standard Pose Estimation results in further reduction of BA error. 

\subsection{Execution time}
In chart \ref{plot:ReconstructionWithoutBA} reconsturction execution time without Bundle Adjustment was presented. Comparing execution time of reconstruction using known rotations and translations from \ref{plot:ExecutionTime} one can see that most of the reconstruction time is used for SIFT correspondences matching. Proper correspondeces matching is bottle-neck in all implemented reconstruction strategies. Reconstruction time with Bundle Adjustment turned on was shown in chart\ref{plot:ReconstructionWithBA}). It turns out that Bundle Adjustment works much faster for strategies that used enhanced initial pair reconstructions and enhanced pose Estimation. As explained earlier such enhanced reconstructions are better organised than standard ones. This results in faster convergence of BA. Sample difference between reconstructed cloud points before and after BA can be seen in \ref{fig:BundleAdjustmentComparison}.
\begin{figure}[b!]
    \centering
    \includegraphics[width=\textwidth]{bundleAdjustmentComparison}
    \caption{3D point clouds before Bundle Adjustment(upper pair) and after(bottom pair) reconstructed using enhanced 8-point initial pair algorithm and enhanced Pose Estimation. Warsaw University of Technology dataset for 1000 SIFT corresponding features(left - front, right - from side)}
    \label{fig:BundleAdjustmentComparison}
\end{figure}
\begin{figure}[h!]
  \begin{center}
    \begin{tikzpicture}
      \begin{axis}[
          width=\linewidth, % Scale the plot to \linewidth
          grid=major, % Display a grid
          grid style={dashed,gray!30}, % Set the style
          xtick = {100,500,1000,5000},
          xlabel=SIFT features number, % Set the labels
          xmode=log,
          log ticks with fixed point,
          ymode=normal,
          ylabel=Execution time(ms),
          legend style={at={(0.5,-0.05)},anchor=north,cells={anchor=west}}, % Put the legend below the plot
        ]
        \addplot[mark=*,blue] table[x=Number of Sift features,y=8-pointInit+NormalPoseEstim,col sep=comma] {ReconstructTotal.csv}; 
        \addplot[mark=*,red] table[x=Number of Sift features,y=Enhanced8-point+NormalPoseEstim,col sep=comma] {ReconstructTotal.csv};
        \addplot[mark=*,orange] table[x=Number of Sift features,y=Enhanced8-point+EnhancedPoseEstim,col sep=comma] {ReconstructTotal.csv}; 
        \addplot[mark=*,pink] table[x=Number of Sift features,y=Alternative3-point+NormalPoseEstim,col sep=comma] {ReconstructTotal.csv};
        \addplot[mark=*,green] table[x=Number of Sift features,y=Alternative3-point+EnhancedPoseEstim,col sep=comma] {ReconstructTotal.csv};
%        \addplot[mark=*,yellow] table[x=Number of Sift features,y=Known Rotations and Translations,col sep=comma] {ReconstructTotal.csv};
        \legend{Standard 8-point initialization + Normal OpenCV Pose Estimation,Enhanced 8-point initialization + normal Pose Estimation,Enhanced 8-point initiazliation + enhanced Pose Estimation, Alternative 3-point initiazliation + normal Pose Estimation, Alternative 3-point initialization + enhanced Pose Estimation}
      \end{axis}
    \end{tikzpicture}
    \caption{Total reconstruction execution time (4 images with resolution 1024x768pixels) in comparison to Sift feature number}
    \label{plot:ReconstructionWithoutBA}
  \end{center}
\end{figure}

\begin{figure}[h!]
  \begin{center}
    \begin{tikzpicture}
      \begin{axis}[
          width=\linewidth, % Scale the plot to \linewidth
          grid=major, % Display a grid
          grid style={dashed,gray!30}, % Set the style
          xtick = {100,500,1000,5000},
          xlabel=SIFT features number, % Set the labels
          xmode=log,
          log ticks with fixed point,
          ymode=normal,
          ylabel=Execution time(ms),
          scaled ticks = false,
          xticklabel style={/pgf/number format/.cd,fixed,precision=5},
          legend style={at={(0.5,-0.05)},anchor=north,cells={anchor=west}}, % Put the legend below the plot
        ]
        \addplot[mark=*,blue] table[x=Number of Sift features,y=8-pointInit+NormalPoseEstim,col sep=comma] {ReconstructBA.csv}; 
        \addplot[mark=*,red] table[x=Number of Sift features,y=Enhanced8-point+NormalPoseEstim,col sep=comma] {ReconstructBA.csv};
        \addplot[mark=*,orange] table[x=Number of Sift features,y=Enhanced8-point+EnhancedPoseEstim,col sep=comma] {ReconstructBA.csv}; 
        \addplot[mark=*,pink] table[x=Number of Sift features,y=Alternative3-point+NormalPoseEstim,col sep=comma] {ReconstructBA.csv};
        \addplot[mark=*,green] table[x=Number of Sift features,y=Alternative3-point+EnhancedPoseEstim,col sep=comma] {ReconstructBA.csv};
%        \addplot[mark=*,yellow] table[x=Number of Sift features,y=Known Rotations and Translations,col sep=comma] {ReconstructTotal.csv};
        \legend{Standard 8-point initialization + normal OpenCV Pose Estimation,Enhanced 8-point initialization + normal Pose Estimation,Enhanced 8-point initiazliation + enhanced Pose Estim, Alternative 3-point initiazliation + normal Pose Estimation, Alternative 3-point initialization + enhanced Pose Estimation}
      \end{axis}
    \end{tikzpicture}
    \caption{Total execution time of reconstruction with Bundle Adjustment turned on(4 images with resolution 1024x768pixels) in comparison to initial SIFT feature set}
    \label{plot:ReconstructionWithBA}
  \end{center}
\end{figure}
\clearpage

\section{Effectivness}
However numerical measuerments do not always results in proper 3D model reconstructions, so some of the intresting 3D reconstraction situations are presented on next 4 pages. In \ref{fig:uni4000Comparison} different Initialization Pair methods were used to reconstruction models for 4000 SIFT features. It can be seen that standard and enhaced methods gave quite good results, which only differ in scale. 3-point algorithm produces quite good models, but slightly distorted due to noise in used rotations. Model reconstructed from known rotations and translations are very distorted, but stil readable. It could be used, when very fast reconstruction is needed. Essential Matrix estimation methods unfortunately failed in terms of relative pose estimation.
\newline
However situation changes a little bit for 400 points. Every algorithm managed to find solution reallt close to optimum. In some cases "Traingulation Test" used in Standard 8-point approach fails and produces unrecognizable model(\ref{fig:FailCaseFundamental}).
\newline
When it comes to Pose Estimations enhancments one can see in \ref{fig:PoseEstimationMethodComparison} that generally initial  rotation and translation help to find better solution, which mainly manifests in less outliers number.
\newline
In figure \ref{fig:UniNone4000} it can be seen, how many outliers using only noisy rotations and translations produces.
\newline
More reconstructed models can be found in [TODO Materials]
\begin{figure}[p]
    \centering
    \includegraphics[width=0.9\textwidth]{uni4000Comparison}
    \caption{Reconstructed models for proposed Initial reconstruction methods and 4000 SIFT Features. From upper left to bottom right following: 1) Standard 8-point, 2) Enhanced 8-point, 3) Alternative 3-point, 4) Known rotations and translations, 5) Standard 5-point, 6) Enhanced 5-point}
    \label{fig:uni4000Comparison}
\end{figure}
\begin{figure}[t!]
    \centering
    \includegraphics[width=0.9\textwidth]{uni400Comparison}
    \caption{Reconstructed models for proposed Initial reconstruction methods and 400 SIFT Features. From upper left to bottom right following: 1) Standard 8-point, 2) Enhanced 8-point, 3) Alternative 3-point, 4) Known rotations and translations, 5) Standard 5-point, 6) Enhanced 5-point}
    \label{fig:uni400Comparison}
\end{figure}
\begin{figure}[p]
    \centering
    \includegraphics[width=0.9\textwidth]{FailCaseFundamental}
    \caption{Fail test case of Standard 8-point triangulation(left) in comparison to fortunate reconstruction}
    \label{fig:FailCaseFundamental}
\end{figure}
\begin{figure}[p]
    \centering
    \includegraphics[width=0.9\textwidth]{PoseEstimationMethodComparison}
    \caption{Pose estimation methods comparison(Views from front and side). Left: Normal Pose Estimation, right: Enhanced Rotation and Translation Pose Estimation}. Less outliers are reconstructed when enhancment is used.
    \label{fig:PoseEstimationMethodComparison}
\end{figure}
\begin{figure}[p]
    \centering
    \includegraphics[height=18cm]{uniNone4000}
    \caption{Reconstruction results from known translations and rotations from different angles. Upper shows front face of building, other from different angles. Many outliers in reconstructed model.}
    \label{fig:UniNone4000}
\end{figure}

% ---------------------------------------------------------------------------
% ----------------------- end of thesis sub-document ------------------------
% ---------------------------------------------------------------------------

%\begin{figure}[h]
%  \begin{center}
%    \begin{tikzpicture}
%      \begin{axis}[
%          width=\linewidth, % Scale the plot to \linewidth
%          grid=major, % Display a grid
%          grid style={dashed,gray!30}, % Set the style
%          xtick = {100,500,1000,5000},
%          xlabel=Sift features number, % Set the labels
%          xmode=log,
%          log ticks with fixed point,
%          ymode=normal,
%          ylabel=Execution time(ms),
%          legend style={at={(0.5,-0.2)},anchor=north,cells={anchor=west}}, % Put the legend below the plot
%        ]
%        \addplot[mark=*,blue] table[x=Number of Sift features,y=8-pointInit+NormalPoseEstim,col sep=comma] {ReconstructInit.csv}; 
%        \addplot[mark=*,red] table[x=Number of Sift features,y=Enhanced8-point+NormalPoseEstim,col sep=comma] {ReconstructInit.csv};
%        \addplot[mark=*,orange] table[x=Number of Sift features,y=Enhanced8-point+EnhancedPoseEstim,col sep=comma] {ReconstructInit.csv}; 
%        \addplot[mark=*,pink] table[x=Number of Sift features,y=Alternative3-point+NormalPoseEstim,col sep=comma] {ReconstructInit.csv};
%        \addplot[mark=*,green] table[x=Number of Sift features,y=Alternative3-point+EnhancedPoseEstim,col sep=comma] {ReconstructInit.csv};
%        \addplot[mark=*,yellow] table[x=Number of Sift features,y=Known Rotations and Translations,col sep=comma] {ReconstructInit.csv};
%        \legend{Standard 8-point initialization + Normal OpenCV Pose Estimation,Enhanced 8-point initialization + Normal Pose Estimation,Enhanced 8-point initiazliation + Enhanced Pose Estim, Alternative 3-point initiazliation + Normal Pose Estimation, Alternative 3-point initialization + Enhanced Pose Estimation, Known Rotations and Translations}
%      \end{axis}
%    \end{tikzpicture}
%    \caption{Execution time of first pair images recontruction(1024x768pixels) in comparison to initial SIFT feature number}
%  \end{center}
%\end{figure}
