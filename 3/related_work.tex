% this file is called up by thesis.tex
% content in this file will be fed into the main document

%: ----------------------- name of chapter  -------------------------
\chapter{Related Work} % top level followed by section, subsection
There are many approaches to the reconstraction problems, from raw one by one pixel analysis for density matchin, high level abstraction of objects recognition and extraction to light and shadows based reconstruction\cite{objectRecAndLoc}\cite{combinedMonoSlam}.
However this document focuses mainly on refining relative structure reconstruction steps. As described in theoretical part in order to estimates first two relative camera positions essential matrix must be decomposed. 
Since basic epipolar geometry equation many scientist introduced a way to solve this linear problems, where the main differences were in terms of accuracy and speed. One of the first was 8-point algorithm\cite{8Point}}, which can be used to compute a fundamental matrix. This is done without any prior knowledge of the scene and camera external and internal parameters. Still to find relative position knowledge of internal camera parameters is needed in order to calculate and decompose Essential Matrix. Later on 5-point algorithm approaches were introduced\cite{Batra5point} \cite{Nister5point}, which used internal camera parameters for the proper essential matrix estimation. M. David Nister in his paper shows that 5-point outperforms almost all similar algorithms in terms of accuracy and speed. Only 8-point algorithm can be competitive, when it comes to forward image sequences. One of the state-of-art real-time and roboust approaches is iterative 5-point algorithm created by Vincent Lui and Tom Drummond\cite{iterative5point}}.

Most of the algorithms are very sensitive to presence of outliers. One of the most common approaches to use of RANSAC modelling\cite{RansacRefine}}, where iteratively  subset of data is chosen to find a solution and then other points are checked, if they also satisfy equation with calculated solution.

Research group from Technishe Univeristät Berlin made an interesting comparison and evaluation of methods, which were discovered by the 2008\cite{HellwichEvaluation}. It turned out that estimation of camera rotation is much more stable than translation calculation. Also there are a lot of ambiguities in terms of choosing the correct solution of epipolar geometry equation. 

In certain situation where external camera parameters as rotation and translation can be measured more accurate algorithms were proposed.  In 2011 D. Scaramuzza from Zurich proposed a 1-point algorithm\cite{1point}, which shows how to describe and use model of camera mounted on a car to enhance 3D reconstruction. Introduced in 2013  4-point algorithm, which uses information of rotation angle in certain axis from additional sensor as show in paper \cite{4point} can outperform even 5-point algorithms. Lately scientists are creating more complex models to estimates relative stereometry. For instance group from Zurich proposed a way to enhance reconstruction with additional 6DOF sensor \cite{robustCameraImu}.

There are also non standard approaches like \cite{lineBasedPose}, where it's shown how to estimates the relative pose from 3  lines with two of the lines parallel and orthogonal to the third. Very accurate estimations also can be achieved for special camera modelsl when there is no rotation\cite{pureTransl}. All these references shows that enhanced models help to achieve often  more accurate or faster solution to standard epipolar geometry problem.

Accuracy and speed are very important, when it comes to create systems capable of augmenting our reality. One of first successful tracking and mapping reconstruction based systems were proposed by research group \cite{ptam}. They showed how two simultaneously working threads can be used to both create model of environment and use this knowledge to apply graphical effects to objects presented on stream camera video. Also some of these concepts where applied already even to robotic vision. Authors showed efficiency of proposed system for robot walking in cluttered indoor workspaces\cite{monoSlam}. There were also approaches of porting reconstruction techniques to mobile solutions\cite{simultanousRecLocMap} \cite{distributedAR} \cite{combinedMonoSlam} \cite{homographyMobile}.

The most important things, which can be concluded are rotation estimation is more stable than translation estimation and the more well described model of setup. Some of researches proposed many interesting scientific papers regarding rotation imporved solution searching \cite{rotationSpaceSearch} \cite{Enqvist10stablestructure}. There are even some approaches with usage of additional IMU accelerometer for reconstruction improvments \cite{robustCameraImu}


% ---------------------------------------------------------------------------
%: ----------------------- end of thesis sub-document ------------------------
% ---------------------------------------------------------------------------