% this file is called up by thesis.tex
% content in this file will be fed into the main document

%: ----------------------- name of chapter  -------------------------
\chapter{Related Work} % top level followed by section, subsection
There are many approaches to the reconstruction problems, starting with the raw one-by-one pixel analysis for density matching, through high level abstraction of objects recognition and extraction and ending with the light and shadows-based reconstruction \cite{objectRecAndLoc}\cite{combinedMonoSlam}.
However this thesis does not focus on high-level abstraction reconstruction, but on refining relative poses estimation steps. As described in the theoretical part, in order to estimate the first two relative camera positions, an essential matrix must be decomposed. 
Since the basic epipolar geometry equation discovery, many scientist have introduced various ways to solve this linear problems, wchich differed primarily in terms of the accuracy and speed. One of the first was the 8-point algorithm \cite{8Point}, which can be used to compute a fundamental matrix. This can be done without any prior knowledge of the scene or camera external and internal parameters. Fundamental matrix has to be converted to essential matrix in order for the relative camera positions to be found and that is why it is necessary to calculate internal camera parameters. Next, the 5-point algorithm approaches were introduced \cite{Batra5point} \cite{Nister5point}. They used internal camera parameters for the proper essential matrix estimation. In his paper M. David Nister showed that the 5-point algorithm outperforms almost all similar algorithms in terms of accuracy and speed. Only the 8-point algorithm can be regarded as equally efficient. One of the state-of-art real-time and robust approaches is the iterative 5-point algorithm created by Vincent Lui and Tom Drummond\cite{iterative5point}.

Most of the algorithms are very sensitive to the occurence of outliers. One of the most common approaches to use to tackle this problem is RANSAC modelling\cite{RansacRefine}.

A research group from Technishe Universität Berlin made an interesting comparison and evaluation of the methods discovered by 2008 \cite{HellwichEvaluation}. It was discovered that estimation of camera rotation is much more stable than translation calculation. Furthermore, there are a lot of ambiguities regarding the choice of the correct solution of epipolar geometry equation. 

In certain situations, where external camera parameters as rotation and translation can be measured, more accurate algorithms were proposed.  In 2011 D. Scaramuzza from Zurich proposed a 1-point algorithm \cite{1point}, which shows how to describe and use a model of a camera mounted on a car to enhance 3D reconstruction. The 4-point algorithm introduced in 2013 using information of rotation angle in certain axis from additional sensor, as shown in paper \cite{4point}, can outperform even the 5-point algorithms. Lately the scientists have been creating more complex models to estimate relative stereometry, e.g. a team from Zurich proposed a way to enhance reconstruction with additional 6DOF sensor \cite{robustCameraImu}.

There are also non standard approaches like \cite{lineBasedPose}, where it is shown how to estimate the relative pose from three lines with two of the lines being parallel and orthogonal to the third one. Very accurate estimations can also be obtained for special camera model cases, when there is no rotation\cite{pureTransl}. All these references show that enhanced models often help to achieve more accurate or faster solutions to standard epipolar geometry problems.

Accuracy and speed constitute significant elements of the process of creating the systems capable of augmenting our reality. One of the first successful tracking and mapping reconstruction-based system was proposed by a research team from Oxford University \cite{ptam}. They showed how two simultaneously working threads can be used to both create a model of the environment being scanned and use this knowledge to apply graphical effects to objects presented on a video stream. Some of these concepts have already been applied to robotic vision. The authors showed efficiency of the proposed system for a robot walking in cluttered indoor workspaces\cite{monoSlam}. There were also approaches of porting reconstruction strategies to mobile devices\cite{simultanousRecLocMap} \cite{distributedAR} \cite{combinedMonoSlam} \cite{homographyMobile}.

The major conclusions to be drawn include the fact that rotation estimation is more stable than translation estimation and that the reconstruction accuracy is improved by a better described model of camera movement. Some of the researches drafted many scientific papers regarding rotation-improved solution space searching \cite{rotationSpaceSearch} \cite{Enqvist10stablestructure}. There are even some approaches involving the use of additional IMU accelerometer for the reconstruction process enhancement \cite{robustCameraImu}.


% ---------------------------------------------------------------------------
%: ----------------------- end of thesis sub-document ------------------------
% ---------------------------------------------------------------------------
