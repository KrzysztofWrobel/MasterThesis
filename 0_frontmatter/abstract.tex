
% Thesis Abstracts -----------------------------------------------------

\selectlanguage{english}
\begin{abstract}
The main subject of this thesis addresses the possibility of enhancing the existing 3D reconstruction algorithms with additional Android sensor fusion data. For this purpose the author of this thesis reviewed the existing solutions for 3D reconstruction and subsequently developed an algorithm for improved relative pose estimation on their basis using sensor fusion data. This thesis introduces the primary fundamentals behind the reconstruction techniques and the problems related to them. Also the related works in the field are discussed herein in order to explain the strengths and weaknesses of the state-of-the-art reconstruction approaches. The proposed enhancements developed using initial rotation estimation in the standard 8-point and 5-point algorithms are explained in detail. Both of them allow for focusing on correcting rotation matrix error instead of calculating the rotation itself. This, in turn, serves for reducing ambiguity of the decomposition of essential matrix to the relative rotation and translation of the cameras. This study also presents the primary idea behind the implementation of the 3-point algorithm for translation. The author also discusses the functioning of the \textit{''Sensor Enhanced Image Camera''} Android application and \textit{''Enhanced 3D Reconstructer''} desktop application implemented by him.
The three proposed initial pair reconstruction methods were evaluated, which showed that each algorithm is able to improve certain aspects of the 3D reconstruction. Different reconstruction strategies were evaluated. As a result it was discovered that the enhanced structure computation can be faster and more accurate than the standard methodology. Using initial rotation and translation pose estimation results in significantly faster convergence and better error reduction with Bundle Adjustment. This thesis covers selected notions of the Structure from Motion, therefore plans for future development were established.
\end{abstract}

\cleardoublepage
\selectlanguage{ngerman} 
\begin{abstract}
Das Hauptthema dieser Arbeit befasst sich mit der M{\"o}glichkeit der Verbesserung der vorhandenen 3D-Rekonstruktionsalgorithmen mit zus{\"a}tzlichen Android Sensorfusion Daten. Die wichtigsten Grundlagen hinter der Rekonstruktionstechniken und die damit verbundenen Probleme sind, um den Leser zu diesen Themen vorstellen erl{\"a}utert. {\"A}hnliche Arbeiten sind erforderlich, um die Festigkeiten und Schw{\"a}chen der state-of-art Rekonstruktion Ansätze erkl{\"a}ren diskutiert. Die vorgeschlagenen Verbesserungen mit Anfangsrotation in den Standard 8-Punkte und 5-Punkte-Algorithmen werden im Detail erkl{\"a}rt. Beide von ihnen k{\"o}nnen zur Fokussierung auf die Rotationsmatrix Fehlerkorrektur anstelle der Berechnung des Rotation selbst. Dies wiederum serviert zur Reduzierung der Mehrdeutigkeit der Essential-Matrix Zersetzung zu der relativen Rotation und Translation der Kameras. Die Hauptidee hinter der Umsetzung des 3-Punkt-Algorithmus f{\"u}r die Translation ist ebenfalls beschrieben. Die wichtigsten Aspekte der implementierten \textit{''Der Sensor Verbesserte Bildkamera''} Android-Anwendung und \textit{''Verbesserte 3D Reconstructer''}  werden diskutiert.
Die drei vorgeschlagenen ersten Paar Rekonstruktionsmethoden werden bewertet, die zeigten, dass jeder Algorithmus kann bestimmte Aspekt der 3D-Rekonstruktion zu verbessern. Verschiedene Rekonstruktionsstrategien werden bewertet. Als Ergebnis wurde festgestellt, dass die verbesserte StrukturBerechnung kann schneller und genauer als die Standardmethode sein. Mit  anfänglicher Rotation und Translation  Posenschätzung resultiert in signifikant schneller Konvergenz und bessere Fehlerreduktion mit B{\"u}ndelausgleichung. Diese Diplomaarbeit umfasst nur einige Vorstellungen von der Struktur der Bewegung, daher plant f{\"u}r die zuk{\"u}nftige Entwicklung entstanden.
\end{abstract}
\cleardoublepage
\selectlanguage{english}
