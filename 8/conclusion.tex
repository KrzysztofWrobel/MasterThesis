% this file is called up by thesis.tex
% content in this file will be fed into the main document

\chapter{Conclusion} % top level followed by section, subsection
This chapter describes the whole work performed during master thesis, discusses encountered problems and proposes ideas for future work.

% ----------------------- contents from here ------------------------
\section{Summary}
The idea of enhancing standard 3D reconstruction algorithms with Android sensor fusion data was concepted, implemented and verified in few different version. At the beginning of this thesis only alternative 3-point version was planned to be made. However it didn't produce results of enough accuracy. After few iterations of implementation and testing 3-point algorithm turned out that basing solely on rotation from Android sensor fusion is not enough. That was when Enhanced 8-points and 5-points versions were designed. In addition author proposed different reconstractiun strategies, which can be used depending on accuracy and speed trade-off. They were built on personal observations of resulted reconstructions and inspired by Related Work papers.
In order to acquire datasets for algoritms evaluation Android app called "Sensor Enhanced Images Camera" was created. When picture is captured it automaticaly stores actual device rotation information and proposed heuristcally estimated translation. 
To evaluate the proposed methods and collected datasets Desktop application called "Enhanced 3D reconstructor" was implemented.
It can be used in 2 different modes:
1) Efficiency testing - to comaprison of Samson Error and visual estimation of calculated epipolar lines.   
Proposed standard 8-point algorithm improvments allows to unambiguously calculate proper rotation and translation. Enhanced 5-point algorithm improvements resulted in better accuracy than standard 5-point algorithm in terms of Sampson Error. Execution time of enhanced reconstruction methods is generally bigger than standard ones. 
Using initial rotation and translation estimation in Pose Estimation results in better reconstruction accuracy especially in terms of outliers removal and Bundle Adjustment convergence speed. 
In general, Bundle Adjustment of sensor enhanced reconstruction results in bigger error reduction and shorter execution time in comparison to refining standard ones. One of the bottle-neck in proposed reconstruction process was correspondences matching time. 
Using only sensor data is not enough to create accurate 3D models. However it's possible to find recognizable model in reconstructed 3D cloud points. The only problem is how to distuinguish recognizable model from reconstracted from outliers uncorrect 3D points. It turnes out that estimating rotation error matrix ($R_{error}$) is much accurate and proposed Rodrigues decomposition and its rounding, such as diagonal of $R_{error}$ consists of diagonal identity matrix, results in better accuracy of 3D reconstruction and unambigously defines proper camera decomposition. Heuristical movement estimation is not very accurate and does not have major impact on reconstroctiun process. Finally proposed 3-point algorithm for translation estimation allows for faster and quite accurate structure recreation.
\section{Dissemination}
Up to this point no one used implemented applications. However Android app can be useful for others 3D reconstruction researches and it is planned to be published to Google Play Store once most needed improvements are made and properly tested. "Enhanced 3D reconstructer" has been already published to GitHub as open-source project distrubuted on LGPL license[reference].
\section{Problems Encountered}
Most of the problems were related to bugs, which appeared during implementation of proposed algorithms and adaptation of 3rd party libraries. Android API allows user to get rotation quaternion and a way to decompose it, but it doesn't explain how it's calculated. Decomposition of rotation matrix to euclidan angles and their composition needs to be done in the same order. Some tests were conductued in order to figure it proper rotation matrix composition. All of these problems were fortunately resolved. It turned out that using pose estimation insted of homography merging was not the very best option. Relaying on pose estimation produced very small amount of points and sometimes reconstruction stoped only after few images analysis.
\section{Future work}
Firstly it would be good to check, how the homography merging approach would influence an accuracy of 3D reconstruction. Secondly 
 other correspondence matching approach should be tested. For example optical flow estimation using video sequences. This would both allow for very quick relative pose estimation and later could be used for dense model reconstruction. All of used librieries are also available or can be ported to Android, so it would be good to see is it possible to achieve a real-time tracking and mapping (similar to \cite{ptam}).



% ---------------------------------------------------------------------------
% ----------------------- end of thesis sub-document ------------------------
% ---------------------------------------------------------------------------