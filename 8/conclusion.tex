% this file is called up by thesis.tex
% content in this file will be fed into the main document

\chapter{Conclusion} % top level followed by section, subsection
This chapter describes the work performed in the course of investingating the master thesis subject, discusses the problems encountered and proposes ideas for future development.

% ----------------------- contents from here ------------------------
\section{Summary}
The idea of enhancing standard 3D reconstruction algorithms with Android sensor fusion data was conceptualised, implemented and verified in a few different versions. At the beginning, only the alternative 3-point version was planned to be implemented for the purpses of this thesis. However, it did not produce results of enough accuracy. After few iterations of implementation and testing of the 3-point algorithm it was concluded that it is not enough to base the reconstruction solely on the data of the rotation from Android sensor fusion. That was when the enhanced 8-point and 5-point versions were designed. In addition, the author proposed different reconstruction strategies, which can be used depending on accuracy and speed trade-offs preferred. They were built on personal observations of the reconstructions performed and inspired by related works in the field.
In order to acquire datasets for algorithm, an evaluation Android application named "Sensor Enhanced Images Camera" was developed. Upon capturing the picture it automaticaly stores the current device's rotation information and proposed heuristcally estimated translation. 
To evaluate the proposed methods and collected datasets a desktop application named "Enhanced 3D reconstructor" was implemented.
It can be used in two different modes:
1) Efficiency testing - for comparing  Samson Error and visual estimation of calculated epipolar lines.   
The proposed enhancements to the standard 8-point algorithm allow to unambiguously calculate the proper rotation and translation. Application of the enhanced 5-point algorithm resulted in better accuracy than in the case of the standard 5-point algorithm in terms of Sampson Error. Execution time of the enhanced reconstruction methods is generally longer than the standard ones'. 
The use of initial rotation and translation estimation in pose estimation results in a greater reconstruction accuracy, particularly in terms of outliers removal and Bundle Adjustment convergence speed. 
In general, applying Bundle Adjustment of sensor enhanced reconstruction results in greater error reduction and shorter execution time in comparison to the refining standard ones. The major problem, a bottleneck of some sort, of the proposed reconstruction process was the time needed for matching corresponding items. 
The study showed that using the sensor data only is not enough to create accurate 3D models. However, it is possible to find recognizable model among the reconstructed 3D cloud of points. The only problem is how to distuinguish a recognizable model from the one reconstructed from uncorrect 3D points. It turned out that estimating rotation error matrix ($R_{error}$) is quite accurate and useful for that purpose and the proposed Rodrigues decomposition and its rounding, such as diagonal of $R_{error}$ consisting of a diagonal identity matrix, returns better accuracy of 3D reconstruction and unambigously defines the proper camera decomposition. Heuristical movement estimation is not entirely accurate and does not have significant impact on the reconstruction process. Finally, the proposed 3-point algorithm for translation estimation allows for faster and quite accurate recreation of the structure.
\section{Dissemination}
So far no one has used the implemented applications. Nonetheless an Android app can be useful for further 3D reconstruction research and it is planned to be published to Google Play Store once most needed improvements are made and properly tested. "Enhanced 3D reconstructer" has been already published to GitHub as open-source project distrubuted on LGPL license [reference].
\section{Problems Encountered}
The majority of the problems were related to bugs which appeared during implementation of the proposed algorithms and adaptation of the third party libraries. Android API allows a user to get rotation quaternion and a way to decompose it, but it does not explain how it is calculated. Decomposition of rotation matrix to euclidan angles and their composition needs to be done in the same order. Some tests were conducted in order to establish its proper rotation matrix composition. All of these problems were successfully resolved. It turned out that using the pose estimation insted of homography merging was not the optimal solution. Relaying on the pose estimation produced very small amount of points and sometimes reconstruction stoped only after analysing merely a few images.
\section{Future work}
Firstly, it would be useful to establish how the homography merging approach would influence an accuracy of 3D reconstruction. Secondly, 
 other correspondence matching approaches should be tested. An optical flow estimation using video sequences could constitute one example thereof. This would both allow for a very quick relative pose estimation and could be used for dense model reconstruction afterwards. All of the libraries used are available or can be ported to Android, therefore it might be valuable to determine whether it is possible to achieve a real-time tracking and mapping (similar to \cite{ptam}).



% ---------------------------------------------------------------------------
% ----------------------- end of thesis sub-document ------------------------
% ---------------------------------------------------------------------------
