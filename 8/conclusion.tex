% this file is called up by thesis.tex
% content in this file will be fed into the main document

\chapter{Conclusions} % top level followed by section, subsection
This chapter describes the work performed in the course of investigating the master thesis subject, discusses the problems encountered and proposes ideas for the future development.

% ----------------------- contents from here ------------------------
\section{Summary}
The idea of enhancing standard 3D reconstruction algorithms with Android sensor fusion data was conceptualised, implemented and verified in several different versions. The main invention of this thesis was a conception and implementation of improved Essential matrix decomposition by combining initial camera rotation estimation from sensor fusion with specifics of Rodrigues parametrisation for small error correcting angles (See equation \ref{eq:rodiguesError}). \\
At the beginning, only the alternative 3-point version was planned to be implemented for the purposes of this thesis. However, it did not produce results of enough accuracy. After few iterations of implementation and testing of the 3-point algorithm it was concluded that it is not enough to base the reconstruction solely on the rotation data from Android sensor fusion. That was when the enhanced 8-point and 5-point versions were designed, which focused on finding error-correcting rotation matrix of the camera orientation and thus limiting decomposition ambiguities. Decomposition, which is hard to do especially, when there are many outliers established during correspondence matching. \\
The proposed enhancements to the standard 8-point algorithm allow to unambiguously calculate the proper rotation and translation (with eventually reversed baseline, which does not influence shape of reconstructed model). Application of the enhanced 5-point algorithm resulted in better accuracy than in the case of the standard 5-point algorithm in terms of Sampson Error. Execution time of the enhanced reconstruction methods is generally longer than of the standard ones. 
The proposed 3-point algorithm for translation estimation allows only for faster and but not very accurate recreation of the structure. Heuristic movement estimation is not entirely accurate and does not have significant impact on the reconstruction process. \\
In addition, the author proposed different reconstruction strategies for SfM process, which can be used depending on accuracy and speed trade-offs preferred. They were built on personal observations of the reconstructions performed and inspired by related works in the field of 3D reconstruction. \\
In order to acquire datasets for algorithms' evaluation, the Android application named ''Sensor Enhanced Images Camera'' was developed. Upon capturing the picture it automatically stores the current device's rotation information and the proposed heuristically estimated translation. 
To evaluate the proposed methods and collected datasets a desktop application named ''Enhanced 3D reconstructor'' was implemented.
It can be used in two different modes:
\begin{enumerate}
\item Efficiency testing - for comparing  Samson Error and visual estimation of calculated epipolar lines.
\item Reconstruction testing - for different reconstruction strategy testing and BA influence    
\end{enumerate}
 What is more, the use of initial rotation and translation in pose estimation results in a greater reconstruction accuracy, particularly in terms of outliers removal, and an increase in Bundle Adjustment convergence speed. \\
In general, applying Bundle Adjustment to model reconstructed with sensor enhancement results in greater error reduction and shorter execution time in comparison to the models acquired through the standard process. The major problem, a bottleneck of some sort, of the proposed reconstruction process was the time needed for matching corresponding items. \\
To sum it up, in author's research and this document it was determined that usage of Sensor Fusion data for initial rotation computation and estimation of rotation error matrix ($R_{error}$) gives more accurate results and is useful for generating reliable 3D models. The proposed Rodrigues decomposition with its rounding helps resolve ambiguities in decomposition of essential matrix to relative rotation and translation, especially when there are many mismatched corresponding points produced during feature matching phase. 
\section{Dissemination}
So far no one has used the implemented applications. Nonetheless the Android application can be useful for further 3D reconstruction research and it is planned to be published to Google Play Store once most needed improvements are made and it is properly tested. ''Enhanced 3D reconstructer'' has been already published to GitHub as open-source project distributed on LGPL license and can be found here: \cite{website:3dReconstructerMasterThesis}.
\section{Problems encountered}
The majority of the problems were related to bugs which appeared during implementation of the proposed algorithms and adaptation of the third party libraries. Android API allows a user to obtain rotation quaternion and offers a way to decompose it, but it does not explain how it is calculated. Decomposition of rotation matrix to euclidean angles and their composition needs to be done in the same order. Some tests were conducted in order to establish its proper rotation matrix composition. All of these problems were successfully resolved. It turned out that using the pose estimation instead of homography merging was not the optimal solution. Relaying on the pose estimation produced few points and sometimes reconstruction was force-stopped after analysing merely a few images.
\section{Future work}
Firstly, it would be valuable to establish how the homography merging approach would influence the accuracy of 3D reconstruction. Secondly, other correspondence matching approaches should be tested. An optical flow estimation using video sequences could constitute one example thereof. This would both allow for a very quick relative pose estimation and could be used for dense model reconstruction afterwards. Author of this research would like to implement such dense reconstruction techniques and produce nice-looking and textured 3D models with bigger resolution. \\
All of the libraries used are available or can be ported to Android, therefore it might be valuable to determine whether it is possible to achieve a real-time tracking and mapping on this platform (similar to \cite{ptam}).



% ---------------------------------------------------------------------------
% ----------------------- end of thesis sub-document ------------------------
% ---------------------------------------------------------------------------
